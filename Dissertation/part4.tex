\chapter{Доработки для ученого совета МГТУ}\label{ch:ch4}

\section{Оформление разделов и подразделов}\label{sec:ch4/sect1}


biblatex: displaying all authors of multi-author works in the bibliography


Simply set the package option maxbibnames=99 in the preamble.

%\usepackage[maxbibnames=99]{biblatex}





%%%%%%%%%%%%%%%%%%%%%%%%%%%%%%%%%%%%

Отделение точкой номеров разделов и подразделов
%Dissertation/setup.tex#L43:

%\setcounter{headingdelim}{2}


\subsection{Подпараграф \cyrdash{} один}\label{subsec:ch4/sect4/sub1}



%%%%%%%%%%%%%%%%%%%%%%%%%%%%%%%%%%%%%%%%%%%%%%%%%%%%%%%%%%%%%%%%%

LaTeX-шаблон для русской кандидатской диссертации и её автореферата.

%## Особенности
%* Кодировка: UTF-8.
%* Стандарт: ГОСТ Р 7.0.11-2011.
%* Поддерживаемые движки: pdfTeX, XeTeX, LuaTeX.
%* Поддерживаемые реализации библиографии: встроенная на движке BibTeX, BibLaTeX
%на движке Biber.

%[**Примеры компиляции шаблона**](https://github.com/AndreyAkinshin/Russian-Phd-LaTeX-Dissertation-Template/releases/latest).

%[**Установка программного обеспечения и сборка диссертации в файлы PDF**](Readme/Installation.md).

%[**Как писать диссертацию на GitHub?**](Readme/github.md)

Обсуждение

Общие вопросы лучше всего писать в gitter-канал:
%[![Join the chat at https://gitter.im/AndreyAkinshin/Russian-Phd-LaTeX-Dissertation-Template](https://badges.gitter.im/Join%20Chat.svg)](https://gitter.im/AndreyAkinshin/Russian-Phd-LaTeX-Dissertation-Template?utm_source=badge&utm_medium=badge&utm_campaign=pr-badge&utm_content=badge)

%Для отчётов об ошибках и для конкретных пожеланий/предложений лучше всего использовать раздел [Issues](https://github.com/AndreyAkinshin/Russian-Phd-LaTeX-Dissertation-Template/issues).

 Структура

%* [dissertation.tex](dissertation.tex): главный файл диссертации.
%* **[папка Dissertation](Dissertation/):** Структурированная система файлов с
шаблоном диссертации.
%  * **папка images:** Папка для размещения файлов изображений, относящихся только
%  к диссертации.
%  * [setup.tex](Dissertation/setup.tex): Файл упрощённой настройки оформления
%  диссертации.
%* [synopsis.tex](synopsis.tex): главный файл автореферата диссертации.
%* **[папка Synopsis](Synopsis/):** Структурированная система файлов с шаблоном
%автореферата.
%  * **папка images:** Папка для размещения файлов изображений, относящихся
%  только к автореферату диссертации.
%  * [setup.tex](Synopsis/setup.tex): Файл упрощённой настройки оформления
%  автореферата.
%* [presentation.tex](presentation.tex): главный файл презентации.
%* **[папка Presentation](Presentation/):** Структурированная система файлов с
%шаблоном презентации.
%  * **папка images:** Папка для размещения файлов изображений, относящихся
%  только к презентации.
%  * [setup.tex](Presentation/setup.tex): Файл упрощённой настройки оформления
%  презентации.
%* **[папка Documents](Documents/):** Полезные документы (ГОСТ-ы и постановления).
%* **[папка PSCyr](PSCyr/):** Пакет PSCyr + инструкции по установке.
%* **[папка BibTeX-Styles](BibTeX-Styles/):** Подборка русских стилевых пакетов
%BibTeX под UTF-8.
%* **[папка common](common/):** Общие файлы настроек и управления содержанием шаблонов.
%  * [characteristic.tex](common/characteristic.tex): Часть общей характеристики
%  работы, повторяющаяся в диссертации и автореферате.
%  * [concl.tex](common/concl.tex): Заключение. Является общим для автореферата
%  и диссертации (согласно [ГОСТ Р 7.0.11-2011](Documents/GOST%20R%207.0.11-2011.pdf),
%  пункты 5.3.3 и 9.2.3).
%  * [data.tex](common/data.tex): Общие данные (название работы, руководитель,
%  оппоненты, ключевые слова и т. п.).
%  * [packages.tex](common/packages.tex) и [styles.tex](common/styles.tex): Общие
%  пакеты и стили оформления автореферата и диссертации.
%  * [setup.tex](common/setup.tex): Общие настройки автореферата и диссертации.
%  В нём же настраивается выбор реализации библиографии.
%* **[папка biblio](biblio/):** Файлы с библиографией.
%  * [author.bib](biblio/author.bib): Публикации автора по теме диссертации.
%  * [registered.bib](biblio/registered.bib): Зарегистрированные патенты и программы для ЭВМ.
%  * [external.bib](biblio/external.bib): Работы которые ссылается автор.
%* **папка images:** Общие файлы изображений шаблонов.
%  * **папка cache:** Папка прекомпелированных рисунков.
%	* [placeholder.txt](images/cache/placeholder.txt): Файл, необходимый для прекомпиляции
%      рисунков в [overleaf](https://www.overleaf.com/).
%* **папка listings:** Общие файлы листингов.
%* **папка letters:** Файлы генерации конвертов для рассылки автореферата.

Дополнительные файлы:

%* [Makefile](Makefile), [compress.mk](compress.mk), [unix.mk](unix.mk),
%  [windows.mk](windows.mk), [examples.mk](examples.mk), [latexmkrc](latexmkrc): Файлы системы сборки шаблона.
%* [usercfg.mk](usercfg.mk): Пользовательские настройки системы сборки шаблона.
%* [indent.yaml](indent.yaml): Файл настройки форматирования исходного кода для
%  [latexindent](https://www.ctan.org/pkg/latexindent).
%* [.editorconfig](.editorconfig): Файл настройки текстовых редакторов, поддерживающих стандарт
%  [editorconfig](https://editorconfig.org/).
%* [Dockerfile](Dockerfile), [install-dockertex.sh](install-dockertex.sh): Файлы генерации
%  [Docker](https://www.docker.com/) образа для сборки шаблона.
%* [siunitx.cfg](siunitx.cfg): Определения величин SI для библиотеки
%  [siunitx](https://ctan.org/pkg/siunitx).
%* [synopsis_booklet.tex](synopsis_booklet.tex),
% [presentation_booklet.tex](presentation_booklet.tex): Файлы генерации печатных версий
% автореферата и презентации.
%* [presentation_handout.tex](presentation_handout.tex): Файл генерации раздаточных материалов из
%  презентации с добавлением подписей под слайдами.
%* [tikz.tex](tikz.tex): Файл изолированной сборки векторной графики [tikz](https://www.ctan.org/pkg/pgf).

 Дополнительная полезная информация

%* [Оформление библиографии](Readme/Bibliography.md)
%* [Как вносить правки в проект](CONTRIBUTING.md)
%* [Полезные ссылки](Readme/Links.md)
%* [Шаблон в галерее шаблонов ShareLaTeX](https://www.sharelatex.com/templates/thesis/russian-phd-latex-dissertation-template) (очень старая версия).

Благодарности

%* Большое спасибо Юлии Мартыновой за [оригинальный вариант шаблона](http://alessia-lano.livejournal.com/4267.html).
%* Большое спасибо [dustalov](https://github.com/dustalov),
%[Lenchik](https://github.com/Lenchik), [tonkonogov](https://github.com/tonkonogov)
%за значительный вклад и обсуждения.
%* Спасибо [storkvist](https://github.com/storkvist), [kshmirko](https://github.com/kshmirko),
%[ZoomRmc](https://github.com/ZoomRmc), [tonytonov](https://github.com/tonytonov),
%[Thibak](https://github.com/Thibak), [eximius8](https://github.com/eximius8),
%[Nizky](https://github.com/Nizky) за полезные правки и замечания.

Лицензия

%CC BY 4.0

Поэтому можно модифицировать и использовать шаблон любым образом, при
условии сохранения авторства на шаблон оформления диссертации в
формате LaTeX (в виде списка авторов в настоящем файле).  При этом не
накладывается никаких ограничений на текст диссертации, все права на
содержательную часть диссертации остаются за её автором.  В том числе,
если в тексте возникает раздел благодарностей (например, научному
руководителю за умелое руководство, коллегам за помощь в работе и
т.д.), то надо ли выносить авторам шаблона
*Russian-Phd-LaTeX-Dissertation-Template* благодарность за помощь в
оформлении диссертации или нет - решает сам диссертант. Использвание
шаблона не накладывает никаких ограничений на использование итоговых
файлов (например, PDF с готовой диссертацией или авторефератом),
т.е. никак не регулирует то, как они распространяются, копируются,
модифицируются и т.д.










\clearpage
