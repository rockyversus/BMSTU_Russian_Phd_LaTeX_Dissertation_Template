%==========================Заголовок

\section{Отзыв научного руководителя диссертационной работы \thesisAuthorLastNameFromFull\ <<{\thesisTitle}>> представленной на соискание ученой степени кандидата технических наук по специальности \thesisSpecialtyNumber\ --- <<\thesisSpecialtyTitle>>}


\vspace{30pt} %число перед fill = кратность относительно некоторого расстояния fill, кусками которого заполнены пустые места

Ступников В.В. окончил \thesisOrganizationShort по специальности «Машины и технология обработки металлов давлением» в 2006 году. Параллельно с обучением в МГТУ Н.Э. Баумана с 2004 года Ступников~В.~В. начал работать на кафедре <<Технологии обработки материалов>> МГТУ им. Н.~Э.~Баумана и в лаборатории горячей обработки материалов. Во время обучения в ВУЗе проявил склонность к аналитической, исследовательской работе, принимал участие в Восьмой Всероссийской конференции молодых ученых и специалистов в 2015~г., финалист конкурса УМНИК в 2016 году. 

Зачислен в очную аспирантуру ОАО НИАТ в октябре 2006 года, которую закончил в сентябре 2009 года. За время обучения в аспирантуре Ступников В.В. проявил себя хорошо подготовленным, сложившимся инициативным исследователем, способным самостоятельно решать сложные научно-технические задачи.

В период 2011---2014~г. работал в \thesisOrganizationShort\ в НОЦ <<НМКН>> в позиции старшего инженера, административным руководителем научно-технических проектов, продемонстрировал высокую работоспособность, целеустремленность, стремление довести любую работу до конечного результата. В это время была начата работа над диссертационной работой. Во время этой работы Ступников~В.~В. продемонстрировал готовность к поиску альтернативных способов получения и обработки экспериментальных данных, что в конечном итоге является залогом его успешной научно-исследовательской работы. Принимал активное участие в выполнении ряда государственных контрактов в рамках Федеральной целевой программы <<Научные и научно-педагогические кадры инновационной России>>, помогал студентам организации учебного процесса при выполнении курсовых и дипломных работ.

В настоящее время работает в Сколковском институте Науки и Технологий на должности руководителя рабочей группы по направлению <<Машиностроение>>. Выполняет все виды учебной нагрузки---читает лекции, проводит семинарские занятия и лабораторные работы, руководит курсовым и дипломным проектированием, научной работой студентов, бакалавров и магистрантов на русском и английских языках.

Выполненная диссертационная работа посвящена решению актуальной научно-технической задачи---{\MakeLowercase\thesisTitle}.

Достоверность полученных результатов обеспечивается комплексным подходом к выполнению работы, сочетанием теоретических и экспериментальных исследований, применением независимых методов с использованием современного экспериментального оборудования и программного обеспечения.

При работе над диссертацией Ступников~В.~В. проявил себя талантливым, инициативным исследователем, способным грамотно формулировать задачи исследования, выбирать оптимальные способы их решения, самостоятельно проводить научно-исследовательскую работу и анализировать полученные результаты. Продемонстрировал глубокие знания в области физических основ пластического деформирования, теоретических основ и практических навыков моделирования технологических процессов пластического формоизменения методом конечных элементов, практические навыки изготовления экспериментальной оснастки и образцов, проведения и обработки результатов эксперимента.

%{\publikacii}

Диссертация написана автором самостоятельно, обладает внутренним единством, содержит новые научные результаты и свидетельствует о личном вкладе автора в науку. В диссертации приведены сведения об использовании полученных автором научных результатов в {\leadingOrganizationTitle}.

 Считаю, что диссертация \thesisAuthorLastNameFrom\ является самостоятельной и логически завершенной научно-квалификационной работой, в которой изложены новые научно-обоснованные технологические решения, позволяющие обеспечить \MakeLowercase{{\thesisTitle}} и полностью отвечает требованиям ВАК к кандидатским диссертациям, а ее автор – Ступников Вадим Владимирович, заслуживает присуждения  степени кандидата технических наук.
 
\vspace{20pt} %число перед fill = кратность относительно некоторого расстояния fill, кусками которого заполнены пустые места
 
\begin{center}
\begin{tabular}[c]{l m{3cm} r}

Научный руководитель---заведующий кафедрой \\ <<Технологии обработки давлением>>\\ \supervisorRegaliaShort & & \supervisorFioShort \\ 
  & \hrulefill &  \\  

\end{tabular}
\end{center}

\begin{flushleft}
\begin{tabular}{l@{~}l@{~}l}
Адрес:  & 105005, г. Москва, 2-я Бауманская ул. д.5. \\
Телефон:    & 8(499)~263-69-01 \\
Email:    & mt6evs@yandex.ru \\
\end{tabular}
\end{flushleft}
 
\clearpage