
\section{Протокол заседания диссертационного совета Д212. по защите диссертации}

Протокол №8
заседания диссертационного совета Д212. по защите диссертации
\thesisAuthorLastNameFromFull\
от 22 октября 2019 года

Состав диссертационного совета утвержден в количестве 24 человек.
Присутствовали на заседании 17 человек.
На заседании присутствовали: председатель совета д.т.н., профессор Колесников Александр Григорьевич (05.02.09), зам. председателя совета д.т.н., доцент Лавриненко Владислав Юрьевич (05.02.09), ученый секретарь совета, к.т.н., доцент Плохих Андрей Иванович (05.16.09), д.т.н., доцент Алдунин Анатолий Васильевич (05.02.09), д.т.н., доцент Баскаков Владимир Дмитриевич (05.02.09), д.т.н., доцент Будиновский Сергей Александрович (05.16.09), д.т.н., профессор Власов Андрей Викторович (05.02.09), д.т.н., профессор Воронцов Андрей Львович (05.02.09), д.т.н., профессор Евсюков Сергей Александрович (05.02.09), д.т.н., старший научный сотрудник Колмаков Алексей Георгиевич (05.16.09), д.т.н., профессор Куксенова Лидия Ивановна (05.16.09), д.т.н., доцент Курганова Юлия Анатольевна (05.16.09), д.т.н., профессор Полянский Владислав Михайлович (05.16.09), д.т.н., профессор Семенов Иван Евгеньевич (05.02.09), д.т.н., профессор Семенов Михаил Юрьевич (05.16.09), д.т.н., профессор Сизов Игорь Геннадьевич (05.16.09), д.т.н., профессор Третьяков Анатолий Федорович (05.16.09).
На заседании отсутствовали: д.т.н., профессор Гневко Александр Иванович (05.16.09), д.т.н., профессор Дмитриев Александр Михайлович (05.02.09), д.т.н., профессор Крапошин Валентин Сидорович (05.16.09), д.т.н., профессор Мельников Эдуард Леонидович (05.02.09), д.т.н., профессор Семенов Борис Иванович (05.16.09), д.т.н., старший научный сотрудник Филатов Александр Андреевич (05.02.09), д.т.н. Юсупов Владимир Сабитович (05.02.09).

ПОВЕСТКА ДНЯ
Защита диссертации \thesisAuthorLastNameFromFull\ на тему: <<\thesisTitle>>, представленной на соискание ученой степени кандидата технических наук по специальности \thesisSpecialtyNumber\ –- <<\thesisSpecialtyTitle>>.
Научный руководитель д.т.н., профессор \supervisorFioShort.
Официальные оппоненты:
1. \opponentOneRegalia\ \opponentOneFio;
2. к.т.н., профессор Шпунькин Николай Фомич.
Ведущая организация: ФГБОУ ВО «Тульский государственный университет».
Слушали: доклад соискателя \thesisAuthorLastNameFrom. по диссертационной работе.
Вопросы по докладу, автореферату и диссертации задавали: 
д.т.н., профессор Власов А.В., д.т.н., доцент Лавриненко В.Ю., д.т.н., профессор Семенов М.Ю.
Слушали: ученого секретаря совета к.т.н., доцента Плохих А.И.
Оглашены следующие документы:
Заключение организации, где выполнялась работа, отзыв ведущей организации и отзывы на автореферат. Все отзывы положительные, имеются отдельные замечания.
Слушали: ответы соискателя \thesisAuthorLastNameFrom. на замечания, содержащиеся в отзыве ведущей организации и в отзывах на автореферат.
Слушали: официального оппонента д.т.н., профессора \opponentOneWhomFioShort. Отзыв по работе положительный, имеются отдельные замечания.
Слушали: ответы соискателя \thesisAuthorLastNameFrom. на замечания официального оппонента д.т.н., профессора \opponentOneWhomFioShort
Слушали: официального оппонента к.т.н., профессора Шпунькина Н.Ф. Отзыв по работе положительный, имеются отдельные замечания.
Слушали: ответы соискателя \thesisAuthorLastNameFrom. на замечания официального оппонента к.т.н., профессора Шпунькина Н.Ф.
Слушали: научного руководителя соискателя д.т.н., профессора Евсюкова С.А.
Вопросов к научному руководителю не было.

В дискуссии по диссертационной работе выступили: 
д.т.н., профессор Полянский В.М., д.т.н., профессор Власов А.В., д.т.н., профессор Семенов М.Ю.

Постановили:
1.	Диссертационным советом сделан вывод о том, что диссертация является научно-квалификационной работой, в которой изложены новые научно-обоснованные технические и технологические решения, направленные на уменьшение утонения стенки тонкостенных труб при гибке, за счет применения градиентного нагрева, имеющие существенное значение для развития технологии изготовления трубопроводов ракетно-космической техники и страны в целом (п. 9 Положения 
о присуждении ученых степеней).
2.	Присудить Долгополову Михаилу Игоревичу ученую степень кандидата технических наук.

Результаты голосования: за – 17, против – нет, недействительных бюллетеней – нет.


\begin{center}
	\begin{tabular}[c]{c m{4cm} l}
		&            &                     \\
		Председатель       &            &                     \\
		диссертационного совета & \hrulefill & \dcHeadFullFIO      \\
		\dcHeadRegalia      &            &                     \\
		Ученый Секретарь     &            &                     \\
		диссертационного совета & \hrulefill & \dcSecretaryFullFIO \\
		\dcSecretaryRegalia   &            &
	\end{tabular}
\end{center}


\clearpage