%==========================Шапка справа Сверху

\hfill\parbox{6cm}{
	\centerline{УТВЕРЖДАЮ}
	\centerline{\leadingOrganizationHeadPos\---}
	\leadingOrganizationTitle
	
	\ktn~\leadingOrganizationHead\
	\linebreak
	{\hbox to 6cm{\hrulefill}}
	{\hbox to 6cm{<<\rule{7mm}{0.4pt}>>\hrulefill~\number\year\,г.}}}
\vspace{0.5cm}


\section{Заключение диссертационного совета}


ЗАКЛЮЧЕНИЕ ДИССЕРТАЦИОННОГО СОВЕТА \defenseCouncilNumber\ НА БАЗЕ
ФЕДЕРАЛЬНОГО ГОСУДАРСТВЕННОГО БЮДЖЕТНОГО 
ОБРАЗОВАТЕЛЬНОГО УЧРЕЖДЕНИЯ ВЫСШЕГО ОБРАЗОВАНИЯ
«МОСКОВСКИЙ ГОСУДАРСТВЕННЫЙ ТЕХНИЧЕСКИЙ УНИВЕРСИТЕТ ИМЕНИ Н.Э. БАУМАНА», МИНИСТЕРСТВА НАУКИ И ВЫСШЕГО 
ОБРАЗОВАНИЯ РОССИЙСКОЙ ФЕДЕРАЦИИ ПО ДИССЕРТАЦИИ
НА СОИСКАНИЕ УЧЕНОЙ СТЕПЕНИ КАНДИДАТА НАУК

%аттестационное дело №_______________________________
решение диссертационного совета от 22 октября 2019 г. № 8

О присуждении Долгополову Михаилу Игоревичу, гражданину РФ, ученой степени кандидата технических наук.
Диссертация на тему: <<\thesisTitle>> по специальности \thesisSpecialtyNumber\ -- <<\thesisSpecialtyTitle>> принята к защите 9 июля 2019 г., протокол № 3 диссертационным советом 
\defenseCouncilNumber\ на базе Федерального государственного бюджетного образователь-ного учреждения высшего образования «Московский государственный техниче-ский университет имени Н.Э. Баумана (национальный исследовательский универ-ситет)», 105005, г. Москва, ул. 2-я Бауманская, 5, приказ о создании совета 
№ 105/нк от 11 апреля 2012 г.
Соискатель \thesisAuthorLastName~\thesisAuthorOtherNames 1987 года рождения.
В 2010 году \thesisAuthorLastName~\thesisAuthorOtherNames окончил ГОУ ВПО «Москов-ский государственный технический университет имени Н.Э. Баумана». Работает 
ведущим инженером-конструктором в ООО «ПТ ГРУПП».
Для сдачи экзамена по специальности \thesisSpecialtyNumber\ «Технологии и машины об-работки давлением» соискатель \thesisAuthorLastName~\thesisAuthorOtherNames был прикреплен в качестве экстерна в \thesisOrganizationShort\ (приказ ректора № 02.06-03/306 
от 19.11.2018). Для подготовки диссертации на соискание ученой степени канди-дата наук без освоения программ подготовки научно-педагогических кадров в ас-пирантуре \thesisOrganizationShort\ соискатель \thesisAuthorLastName~\thesisAuthorOtherNames 
являлся прикрепленным лицом (приказ ректора № 02.09-03/246 от 24.04.2019).
Диссертация выполнена на кафедре технологий обработки давлением ФГБОУ ВО \thesisOrganizationShort.
Научный руководитель – \supervisorRegaliaShort\ \supervisorFio, ФГБОУ ВО \thesisOrganizationShort, кафедра 
технологий обработки давлением, заведующий кафедрой.

Официальные оппоненты:
\opponentOneFio, доктор технических наук, профессор, не работа-ет;
Шпунькин Николай Фомич, кандидат технических наук, профессор, ФГБОУ ВО «Московский политехнический университет», кафедра обработки ма-териалов давлением и аддитивных технологий, профессор
дали положительные отзывы на диссертацию.
Ведущая организация ФГБОУ ВО «Тульский государственный универси-тет», г. Тула в своем положительном отзыве, подписанном зам. заведующего ка-федрой механики пластического формоизменения д.т.н., профессором Панфило-вым Геннадием Васильевичем и к.т.н., доцентом кафедры механики пластическо-го формоизменения Пасынковым Андреем Александровичем указала, что пред-ставленная диссертация выполнена на актуальную тему и содержит технологиче-ские решения, имеющие существенное значение для развития машиностроитель-ной отрасли страны.
Соискатель имеет 15 научных работ общим объемом 4,3 печ. л., в том чис-ле 6 в изданиях, входящих в Перечень рецензируемых научных изданий ВАК РФ 
и 5 патентов РФ.
Наиболее значимые научные работы по теме диссертации:
1. Долгополов М.И. Оценка утонения тонкостенных труб при гибке с уз-козо-нальным индукционным нагревом // Технология машиностроения. 2016. № 11. 
С. 9-14. (0,9 п.л.).
2. Долгополов М.И., Евсюков С.А. Исследование гибки тонкостенных труб с узкозональным градиентным нагревом // Заготовительные производства маши-ностроения. 2019. № 2. С. 61-65. (0,6 п.л./ 0,4 п.л.).
Соискателем в данной работе разработана математическая модель с исполь-зованием метода конечных элементов, позволяющая определять параметры напряженно-деформированного состояния материала тонкостенных труб при гиб-ке с узкозональным градиентным нагревом.
3. Долгополов М.И., Евсюков С.А. Определение утонения тонкостенных труб при гибке с узкозональным градиентным нагревом Известия ТулГУ. Техни-ческие науки. 2019. № 5. С. 345-354. (0,7 п.л./ 0,5 п.л.).
Соискателем в данной работе разработана математическая модель с исполь-зованием инженерного метода, позволяющая определять утонение стенки тонко-стенных труб при гибке с градиентным нагревом, в зависимости от температурных параметров гибки.
4. Трубогибочный станок: патент 169825 РФ / М.И. Долгополов [и др.]; за-явл. 30.06.2016; опубл. 03.04.2017. Бюлл. №10.
5. Устройство для зонального нагрева: патент 144696 РФ / М.И. Долгопо-лов [и др.]; заявл. 28.03.2014; опубл. 27.08.2014. Бюлл. №24.
На диссертацию и автореферат поступили отзывы: 
1. ФГБОУ ВО «СибАДИ», подписан д.т.н., проф. кафедры эксплуатации и сервиса транспортно-технологических машин и комплексов в строительстве Кузнецовой В.Н. Отзыв положительный с замечаниями:
а) Осталось неясным, отличаются ли допущения и граничные условия, принятые автором диссертации для расчета утонения стенки трубы при гибке с поворотным рычагом, проведенные с использованием «инженерного» метода, и для моделирования процесса гибки трубы в САПР «ANSYS» в среде «Workbench».
б) Не указано, в чем заключается теоретическая значимость выполненных автором работы исследований.
2. «ВМЗ» – филиал АО «ГКНПЦ им. М.В. Хруничева», подписан замести-телем директора завода – главным инженером Кольцовым В.И. и к.т.н., главным технологом Юхневичем С.С. Отзыв положительный с замечанием:
Как недостаток необходимо отметить недостаточность освещения вли-яния уменьшения утонения и применения градиентного нагрева на эксплуатаци-онные свойства получаемых трубопроводов и отсутствие информации о ре-жимах нагрева труб из различных материалов.
3. НИТУ «МИСиС», подписан д.т.н., проф. кафедры обработки металлов давлением Галкиным С.П. Отзыв положительный с замечаниями:
а) Из содержания автореферата не представляется возможным устано-вить происхождение уравнения равновесия, представленных на стр. 6.
б) В автореферате не указано в каком интервале радиусов гиба справед-ливо приведенное на стр. 7 утверждение о том, что нейтральная линия при гибке совпадает со средней линией трубы.
4. ФГБОУ ВО «БГТУ», подписан к.т.н., доц., заместителем первого прорек-тора по учебной работе Василенко Ю.В. Отзыв положительный с замечанием:
На фотографиях образцов изогнутых труб, приведенных в диссертацион-ной работе, видно заметное уменьшение высоты поперечных сечений, данные ее измерения не приведены. Если оно не проводилось, то это явное упущение. Хотя в работе соискателя центральное место занимает утонение стенки трубы, но игнорировать овальность сечений при этом не следовало.
5. НИУ «БелГУ», подписан д.т.н., проф. кафедры материаловедения и нано-технологий Афониным А.Н. Отзыв положительный с замечаниями:
а) В качестве последнего пункта научной новизны работы следует при-знать не саму конечноэлементную модель процесса гибки, а выявленные с ее по-мощью закономерности.
б) Из работы не ясно, какая модель реологических свойств материала за-готовки была принята при конечноэлементом моделировании.
6. ООО «ИЛМиТ», подписан к.т.н., руководителем проекта департамента деформируемых сплавов и композиционных материалов Легких А.Н. и д.т.н., ге-неральным директором Дьяченко А.Н. Отзыв положительный с замечанием:
В математической модели материала принимается жестко-пластическая модель материала. Желательно оценить влияние упрочнения ме-талла на процесс образования утонения.
7. ООО «КванторФорм», подписан, к.т.н., инженером отдела технической поддержки Харсеевым В.Е. и к.т.н., генеральным директором Стебуновым С.А. Отзыв положительный с замечаниями:
а) Разработанная модель гибки в комплексе Ansys рассматривает дефор-мирование только зоны нагрева, что позволяет повысить скорость расчетов, однако не позволяет учесть влияние овализации и угла гибки на утонение.
б) В тексте автореферата не приведено обоснование выбора температу-ры нагрева при моделировании методом конечных элементов.
8. Самарского университета, кафедра обработки металлов давлением, под-писан д.т.н., проф. Поповым И.П. и академиком РАН, д.т.н., проф., зав. кафедрой Гречниковым Ф.В. Отзыв положительный с замечанием:
В автореферате диссертационной работы отсутствуют графические зависимости, иллюстрирующие результаты проведенных исследований, что не-сколько затрудняет их восприятие и анализ.
Выбор официальных оппонентов и ведущей организации обосновывается их известностью своими достижениями в данной отрасли науки, наличием публи-каций по схожей проблематике, отсутствием совместных с соискателем исследо-ваний и печатных работ.
Диссертационный совет отмечает, что на основании выполненных со-искателем исследований:
предложена новая математической модель процесса гибки тонкостенных труб с постоянной температурой нагрева и с применением градиентного нагрева, учитывающая смещение нейтральной линии;
определены аналитические зависимости, позволяющие определить утонение стенки в зависимости от радиуса изгиба и функции распределения напряжений те-кучести по сечению трубы;
разработана конечно-элементная модель процесса гибки тонкостенных труб с градиентным нагревом, позволяющая определить утонение и параметры напряженно-деформированного состояния материала трубы в зависимости от геометрических размеров сечения, свойств материала и функции распределения температуры нагрева по сечению трубы.


Теоретическая значимость исследования обоснована тем, что:
определена закономерность смещения нейтральной линии напряжений и деформаций при гибке тонкостенных труб с узкозональным нагревом;
получены зависимости, позволяющие с точностью до ±3,5% определить де-формации стенки тонкостенных труб от силовых параметров используемой тех-нологической схемы гибки с узкозональным нагревом; 
раскрыто влияние распределения температуры нагрева по сечению тонко-стенной трубы при гибке с градиентным нагревом на изменение величины утоне-ния стенки.
Значение полученных соискателем результатов исследования для практики подтверждается тем, что: 
разработанный специальный станок модели СГИН-120 для гибки с узкозо-нальным индукционным нагревом тонкостенных трубопроводов агрегатов изде-лий ракетно-космической техники, реализующий разработанную технологию гиб-ки с градиентным нагревом, изготовлен и испытан на ФГУП «НПО «Техномаш»;
предложенная математическая модель напряженно-деформированного со-стояния материала при гибке труб с узкозональным нагревом, а также разрабо-танная на ее основе технология гибки с применением градиентного нагрева, ис-пользована ФГУП «НПО «Техномаш» при разработке технологии изготовления трубопроводов жидкостных ракетных двигателей;
предложена методика контроля деформаций стенки при гибке тонкостен-ных труб с градиентным нагревом на основании данных измерения температуры нагрева в четырех точках на поверхности трубы;
разработана конструкция и алгоритм работы устройства для создания и контроля градиентного нагрева, позволяющего снизить утонение при гибке тон-костенных трубопроводов с узкозональным индукционным нагревом;
достигнуто уменьшение утонения 6% при гибке с градиентным нагревом труб 36х2 мм из стали ВНС16 на станке СГИН-120, что позволяет снизить массу трубопровода на 8%.
Результаты диссертационного исследования могут быть использованы на предприятиях, использующих или собирающихся внедрять гибку тонкостенных труб с узкозональным индукционным нагревом, в частности, АО «НПО Энерго-маш», «Воронежский механический завод», АО «РКЦ «Прогресс».
Оценка достоверности результатов исследования выявила:
эксперименты проведены на разработанном и испытанном станке модели СГИН-120, измерения проводились с помощью сертифицированных измеритель-ных приборов, полученные результаты согласуются с результатами эксперимен-тов других авторов;
разработанная теория построена на известных положениях «инженерно-го» метода и математической теории пластичности;
математические расчеты выполнены с использованием апробированных программных комплексах ANSYS и Mathcad.
Личный вклад соискателя состоит в: 
непосредственном участии автора в постановке целей и задач исследования, формулировке основных допущений и выработке методики поиска решений, ана-лизе полученных результатов; 
лично соискателем выполненном аналитическом обзоре научно-технической информации по теме исследования в российской и зарубежной лите-ратуре; разработанной математической модели процесса формоизменения; подго-товленных и проведенных экспериментальных исследованиях; разработанной конструкции станка для гибки труб. 
Диссертация охватывает основные вопросы поставленной научной задачи и соответствует критерию внутреннего единства, что подтверждается наличием по-следовательного плана исследования и взаимосвязи выводов.
Диссертационным советом сделан вывод о том, что диссертация является научно-квалификационной работой, в которой изложены новые научно-обоснованные технические и технологические решения, направленные на умень-шение утонения стенки тонкостенных труб при гибке, за счет применения гради-ентного нагрева, имеющие существенное значение для развития технологии изго-товления трубопроводов ракетно-космической техники и страны в целом (п. 9 Положения 
о присуждении ученых степеней).
На заседании 22 октября 2019 г. диссертационный совет принял решение присудить Долгополову М.И. ученую степень кандидата технических наук.
При проведении тайного голосования диссертационный совет в количестве 17 человек, из них 8 докторов наук по специальности \thesisSpecialtyNumber\ – Технологии 
и машины обработки давлением рассматриваемой диссертации, участвовавших 
в заседании, из 24 человек, входящих в состав совета, проголосовали: «за» – 17, 
«против» – нет, недействительных бюллетеней – нет.


\begin{center}
	\begin{tabular}[c]{c m{4cm} l}
		&            &                     \\
		Председатель       &            &                     \\
		диссертационного совета & \hrulefill & \dcHeadFullFIO      \\
		\dcHeadRegalia      &            &                     \\
		Ученый Секретарь     &            &                     \\
		диссертационного совета & \hrulefill & \dcSecretaryFullFIO \\
		\dcSecretaryRegalia   &            &
	\end{tabular}
\end{center}


\clearpage

