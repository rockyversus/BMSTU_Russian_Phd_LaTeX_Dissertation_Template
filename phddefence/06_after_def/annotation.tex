
\section{Аннотация}

Целью работы является уменьшение утонения стенки при гибке тонкостенных труб с узкозональным нагревом за счет применения градиентного нагрева. Сущность градиентного нагрева заключается в уменьшении температуры внешней стороны гиба и увеличении температуры внутренней стороны, что приводит к смещению нейтральной линии деформаций и уменьшению деформаций утонения. Для достижения цели был проведен анализ современного состояния технологии гибки труб с узкозональным нагревом, была разработана математическая модель образования утонения и определены общие закономерности деформации стенки, разработана конечно-элементная модель процесса гибки тонкостенных труб с градиентным нагревом, проведена экспериментальная проверка разработанных моделей, разработана методика автоматизированного контроля градиентного нагрева, а также конструктивная схема и алгоритм работы устройства, реализующего разработанную методику.
189 страниц, 114 рисунков, 4 таблицы, библиография из 114 наименований.

Ключевые слова: гибка; индукционный нагрев; градиентный нагрев; утонение.

УДК \thesisUdk

Код тематической рубрики 55.16.22

Код международного классификатора 02.00.00


\clearpage