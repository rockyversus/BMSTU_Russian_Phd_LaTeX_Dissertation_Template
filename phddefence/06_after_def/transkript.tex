\section{стенограмма N8}
заседания диссертационного совета \defenseCouncilNumber\
при \thesisInOrganization\
по присуждению ученых степеней

г. Москва	22 октября 2019 г.

На заседании присутствовали: председатель совета д.т.н., профессор 
Колесников Александр Григорьевич (05.02.09), зам. председателя совета д.т.н., 
доцент Лавриненко Владислав Юрьевич (05.02.09), ученый секретарь совета, к.т.н., доцент Плохих Андрей Иванович (05.16.09), д.т.н., доцент Алдунин Анатолий 
Васильевич (05.02.09), д.т.н., доцент Баскаков Владимир Дмитриевич (05.02.09), д.т.н., доцент Будиновский Сергей Александрович (05.16.09), д.т.н., профессор 
Власов Андрей Викторович (05.02.09), д.т.н., профессор Воронцов Андрей Львович (05.02.09), д.т.н., профессор Евсюков Сергей Александрович (05.02.09), д.т.н., 
старший научный сотрудник Колмаков Алексей Георгиевич (05.16.09), д.т.н., 
профессор Куксенова Лидия Ивановна (05.16.09), д.т.н., доцент Курганова Юлия 
Анатольевна (05.16.09), д.т.н., профессор Полянский Владислав Михайлович (05.16.09), д.т.н., профессор Семенов Иван Евгеньевич (05.02.09), д.т.н., профессор 
Семенов Михаил Юрьевич (05.16.09), д.т.н., профессор Сизов Игорь Геннадьевич (05.16.09), д.т.н., профессор Третьяков Анатолий Федорович (05.16.09).
На заседании отсутствовали: д.т.н., профессор Гневко Александр Иванович (05.16.09), д.т.н., профессор Дмитриев Александр Михайлович (05.02.09), д.т.н., 
профессор Крапошин Валентин Сидорович (05.16.09), д.т.н., профессор Мельников Эдуард Леонидович (05.02.09), д.т.н., профессор Семенов Борис Иванович (05.16.09), д.т.н., старший научный сотрудник Филатов Александр Андреевич (05.02.09), д.т.н. Юсупов Владимир Сабитович (05.02.09).

Председатель совета д.т.н., профессор Колесников А.Г. сообщает: 
Кворум есть. Присутствуют 17 человек из 24 членов совета, в том числе докторов наук по специальности 05.02.09 рассматриваемой диссертации – 8 человек.

ПОВЕСТКА ДНЯ:
Защита диссертации \thesisAuthorLastNameFromFull\ на тему: <<\thesisTitle>>, представленной на соискание ученой степени кандидата технических наук по специальности \thesisSpecialtyNumber\ –- <<\thesisSpecialtyTitle>>.
Официальные оппоненты:
1. \opponentOneRegalia\ \opponentOneFio;
2. \opponentTwoRegalia, проф. Шпунькин Николай Фомич.
Ведущая организация: ФГБОУ ВО «Тульский государственный университет».
Ученый секретарь к.т.н., доцент Плохих А.И. оглашает биографические данные о соискателе по материалам личного дела и сообщает, что представленные документы соответствуют требованиям ВАК.
Вопросов по биографическим данным нет.

Председатель совета д.т.н., профессор Колесников А.Г. предоставляет слово соискателю Долгополову М.И.
Долгополов М.И. кратко излагает основное содержание диссертационной работы, дает обоснование актуальности темы, приводит результаты исследований, иллюстрируя их графическим материалом, формулирует выводы. Автореферат диссертации имеется у каждого члена совета и в личном деле соискателя.
Слайд 1
Здравствуйте, уважаемый председатель, уважаемые члены диссертационного совета, уважаемые коллеги.
Слайд 2
В последнее время при изготовлении трубопроводов ракетно-космической техники все более широкое применение находят тонкостенные трубы из специальных жаропрочных сплавов. Поскольку данные материалы не предусматривают деформирование в холодном состоянии, то для гибки таких труб необходимо использовать гибку с узкозональным индукционным нагревом, а основным дефектом проявляющемся при данном виде гибки является утонение стенки (то есть уменьшение толщины) на наружной стороне гиба. Наиболее перспективным способом борьбы с данным дефектом является применение градиентного нагрева.
Таким образом целью данной работы является уменьшение утонения стенки при гибке тонкостенных труб с узкозональным нагревом за счет применения градиентного нагрева.
Слайд 3
Рассмотрим саму гибку с узкозональным нагревом. Ее сущность состоит в непрерывно-последовательном создании узкой зоны нагрева на заготовке с последующим ее охлаждением и приложением к данной зоне изгибающего момента.
В настоящее время применяются два способа гибки c узкозональным нагревом: гибка роликом и гибка поворотным рычагом (водилом). 
Основным способом гибки на малые радиусы является гибка поворотным рычагом. В данном случае труба проталкивается через фильеру (или опорные ролики) и через кольцевой индуктор, при этом поворачивая гибочный рычаг. Ось вращения рычага находится в плоскости индуктора. В результате возможно получение гибов высокой геометрической точности.
Важным условием осуществления техпроцесса является создание узкой зоны нагрева. Для этого применяются индукторы особой конструкции. Основными частями такого индуктора являются: непосредственно водоохлаждаемый магнитопровод, расположенное за ним, по ходу движения трубы, кольцо, через которое на поверхность трубы подается вода и кольцо воздушного поддува, установленное позади индуктора и препятствующее попаданию воды в зону нагрева. 
Слайд 4
Основными дефектами, возникающими при гибке, являются: овализация (искажение формы) поперечного сечения, гофрообразование (потеря устойчивости на внутренней стороне гиба) и утонение стенки на наружной стороне гиба.
Способом борьбы с овализацией и гофрообразованием является уменьшение ширины зоны нагрева. Применение индукторов описанной конструкции позволяет создавать зону нагрева шириной в 2-3 толщины стенки, что позволяет свести риск гофрообразования и степень овализации к минимуму. Наиболее существенным дефектом при этом остается утонение стенки, величина которого не зависит от ширины зоны нагрева.
Существует два способа борьбы с утонением: применение подпора и градиентного нагрева. Оба способа основаны на смещении нейтральной линии в сторону зоны растяжения. При гибке с подпором к гибочному рычагу прикладывается дополнительный подпорный момент, направленный противоположно направлению движению трубы, и создающий дополнительные сжимающие напряжения в зоне нагрева. При гибке с градиентным нагревом создается неравномерное температурное поле в зоне нагрева – зона сжатия подогревается, а зона нагрева подхолаживается, что также приводит к смещению нейтральной линии.
В настоящее время, несмотря на значительные исследования, не создано теории и технологических рекомендаций применения гибки с градиентным нагревом. Кроме того, недостаточно исследован и процесс образования утонения и при гибке без градиентного нагрева. 
Слайд 5
Для достижения цели исследования необходимо разработать математическую модель образования утонения. Для этого рассмотрим гибку с постоянной температурой нагрева.
Примем следующие допущения:
- считаем, что металл деформируется только в зоне нагрева;
- пренебрегаем овализацией поперечного сечения;
- считаем, что охлаждение трубы происходит мгновенно;
- принимаем идеальную жестко-пластическая модель материала;
- так как рассматривается гибка тонкостенных труб, то считаем температуру нагрева стенки по глубине постоянной.
%Принятые допущения справедливы при радиусе гиба больше, чем 1,5𝐷.
%Рассмотрим участок зоны нагрева, непосредственно прилегающий к холодной зоне и соответствующий бесконечно малому углу dα. Используем цилиндрическую систему координат. Положению нейтральной линии соответствует угол θ0. На трубу действует изгибающий момент и продольная сжимающая сила. Стенка трубы под действием нагрузки удлиняется и утоняется (в зоне растяжения) и сжимается и утолщается (в зоне сжатия). Это вызывает появление осевых и круговых напряжений, а также осевых и радиальных деформаций. Касательные напряжения отсутствуют.
Значения круговых и осевых напряжений можно найти исходя из полного условия пластичности Губера-Мизеса, а также уравнения Генки-Ильюшина относительно круговых деформаций. Тогда, учитывая, что круговые деформации близки к нулю, а радиальные напряжения много меньше осевых и круговых, получаем следующие выражения для осевых и круговых напряжений. Они постоянны с переменой знака на нейтральной линии.
Слайд 6
Найти утонение и положение нейтральной линии можно исходя из двух условий: условие постоянства объема (которое можно геометрически определить следующим выражением) и условия равновесия граничного сечения, то есть равенства равнодействующей осевых напряжений и продольной силы, выраженного в форме интегрального уравнения из которого можно найти угол нейтральной линии.
Были получены решения данного интегрального уравнения для двух частных случаев: гибки чистым моментом (в этом случае продольная сила равна нулю) и гибки поворотным рычагом (продольная сила определяется отношением изгибающего момента к радиусу гиба).
Значения угла нейтральной линии для гибки чистым моментом показаны на графике, а для гибки поворотным рычагом получилось, что нейтральная линия всегда совпадает со средней. Также получены графики утонения в зависимости от радиуса гиба для данных двух случаев.
Слайд 7
%Рассмотрим гибку с градиентным нагревом Напряженно-деформированное состояние в этом случае отличается наличием касательный напряжений τρθ. Касательные напряжения должны возникать для сохранения сечения в равновесии вследствие различных значений напряжений текучести в зонах растяжения и сжатия.
%В дальнейших расчетах рассматриваем также уравнения равновесия, полное условие пластичности Губера-Мизеса и уравнение Генки-Ильюшина относительно круговых деформаций. В результате получаем выражения для осевых и круговых напряжений. Более подробные расчеты показали, что касательные напряжения τρθ малы и практически не влияют на значения нормальных напряжений. Поэтому для дальнейших расчетов можно принять выражения, аналогичные полученным для гибки с постоянной температурой нагрева и отличающиеся тем, что в данном случае напряжение текучести является не постоянной величиной, а функцией от угла θ.

Слайд 8
Положение нейтральной линии и утонение определяется аналогично исходя из условий сохранения объема и равновесия граничного сечения. В результате получено общее интегральное уравнение для определения угла нейтральной линии в зависимости от функции распределения напряжений текучести.
Данное уравнение было решено для двух частных случаев – линейного и ступенчатого распределения напряжений текучести. При ступенчатом распределении напряжения текучести минимальны в зоне сжатия и максимальны в зоне растяжения с переменой знака на нейтральной линии. Такое распределение недостижимо в реальном техпроцессе, однако дает максимально возможное уменьшение утонения и позволяет оценить общую эффективность метода градиентного нагрева. Линейное распределение больше соответствует реальному техпроцессу.
Далее для простоты используются относительные величины: относительный радиус гиба – отношение радиуса гиба к среднему диаметру трубы и коэффициент градиента напряжений – определяет степень градиентного нагрева и равен отношению максимальных напряжений текучести к минимальным. Эффективность применения градиентного нагрева определяется относительным уменьшением утонения – величиной, показывающей, на сколько уменьшились деформации утонения в результате применения градиентного нагрева, по сравнению с гибкой без применения градиентного нагрева. На слайде показаны зависимости относительного уменьшения утонения от коэффициента градиента напряжений для двух описанных случаев.
Слайд 9
Тем не менее в реальном техпроцессе точное распределение напряжений текучести неизвестно, поэтому необходимо установить общие закономерности. Для этого были выбраны 9 различных функций, характеризующих различные распределения напряжений текучести. Для каждой из этих функций произведен численный расчет и получены следующие результаты: с точностью 10% можно считать, что относительное уменьшение утонения не зависит от радиуса гиба и линейно зависит от коэффициента градиента напряжений. Этот результат описывается следующим выражением через коэффициент K_f, значение которого меняется в пределах от 0 до 1.
Слайд 10
Процесс образования утонения был исследован также с помощью метода конечных элементов. Предложена упрощенная модель, рассматривающая деформирование только зоны нагрева. Действие инструментов заменено граничными условиями. Действие поворотного рычага смоделировано жесткой плоскостью, вращающейся на шарнире, а действие фильеры – закреплением центральной части сечения. 
Принятые допущения аналогичны допущениям в аналитическом расчете, за исключением того, что принята упруго-пластическая модель материала с линейным упрочнением. Моделировалась гибка трубы 36х2 мм из стали AISI 321.

Слайд 11
Вначале рассматривалась гибка при постоянной температуре нагрева. На слайде показано деформированное сечение трубы, а также графики силовых параметров. Значения момента гибки и продольной силы в целом соответствуют аналитическому расчету, с поправкой на упрочнение. Получены значения поперечной силы, которая не рассматривалась в аналитическом расчете.
Слайд 12
Определены значения компонентов тензора напряжений. Основными компонентами, как было определено в аналитическом расчете, являются круговые и осевые напряжения. Их значения также, с поправкой на упрочнение, соответствуют аналитическому расчету. 
Слайд 13
Определена зависимость утонения от радиуса гиба и деформаций стенок. Значения утонения больше аналитических на 10% выше.
Слайд 14
Проведено моделирование при градиентном нагреве. Значения напряжений также соответствуют аналитическому расчету с учетом упрочнения.
Слайд 15
Моделировался нагрев при линейном распределении температуры нагрева. Значения утонения также превосходят аналитические на 10%. Также было определено что, как и в аналитическом расчете, относительное уменьшение утонения не зависит от радиуса гиба и линейно зависит от коэффициента градиента напряжений.
Слайд 16
Аналогичные результаты получены для степенного распределения температуры нагрева с показателем степени 3 и 7. 
Слайд 17
Экспериментальная проверка проводилась на специальном станке для гибки труб модели СГИН-120. Станок позволяет производить гибку как поворотным рычагом, так и роликом с автоматической переналадкой между ними. Станок оснащен системой ЧПУ и имеет 7 управляемых осей. На конструкцию станка получено 4 патента на полезную модель. Станок разработан с учетом проведенных исследований
Слайд 18
Использовались следующие средства измерения. Непосредственные измерения толщины стенки трубы производились ультразвуковым толщиномером с калибровкой на материале трубы по показаниям микрометра. Непосредственные измерения температуры производились ручным пирометром, а также пирометром, встроенным в кольцо индуктора.
Слайд 19
Градиентный нагрев устанавливается раздельным регулированием подачи воды на внутреннюю и наружную стороны гиба. Также станок оснащен системой, автоматически поддерживающей температуру нагрева постоянной, за счет изменения мощности нагрева. Данная система позволила поддерживать температуру с точностью ±200С.
Слайд 20
Проведены эксперименты по гибке поворотным рычагом труб 36х2 мм из стали 12Х18Н9Т. Гибка при различных температурах нагрева показала, что утонение не зависит от температуры нагрева. Были проведены прямые замеры толщины стенки, показавшие, что аналитическая модель определяет утонение с точностью 7%. Аналогичный результат получается для экспериментов W. Zutang и H. Zhong по гибке стальных труб 89х6 и 89х4,5 мм.
Слайд 21
Проведены эксперименты по гибке с градиентным нагревом труб 36х2 мм из стали ВНС-16. Слева показан разрезанный образец трубы. Непосредственные замеры микрометром показали, что снижение утонения составило 6%, при этом с точностью 3% утонение не зависит от угла гиба. 
Гибка на различные радиусы показала, что относительное уменьшение утонения с точностью 15% не зависит от радиуса гиба. Относительное уменьшение утонения в данном случае определялось косвенно, по измерениям системы ЧПУ длины продольной оси. Приведены также данные фирмы Yeoh Hwa (Сингапур), показывающие, что также утонение с точностью 14% не зависит от радиуса гиба.
Таким образом, разработанная теория имеет хорошее совпадение с результатами эксперимента и может быть использована на практике.
Слайд 22
Вкратце о применении градиентного нагрева в ракетно-космической отрасли. Градиентный нагрев позволяет снижать утонение стенки, что позволяет уменьшить массу трубопроводов. Достичь этого можно двумя способами: использовать более тонкостенные трубы или уменьшить радиус гиба. Так полученное уменьшение утонения на 6% позволяет уменьшить массу колена трубопровода на 8%.
Слайд 23
В ходе проведения экспериментов на станке СГИН-120 был выявлен его недостаток, а именно невозможность контролировать распределение температуры нагрева по сечению. Поэтому было разработано устройство, создающее и контролирующее градиентный нагрев в автоматизированном режиме.
Создание градиентного нагрева в устройстве происходит с помощью смещения индуктора в поперечном направлении, а измерение температуры осуществляется в нескольких точках пирометрами, встроенными в кольцо индуктора. Полученные значения интерполируются.
Слайд 24
Были рассмотрены различные способы интерполяции и количество пирометров. Определено, что наилучшие результаты дают 4 пирометра и способ интерполяции Лагранжа. Преимуществом данного способа является то, что он позволяет получить аналитическое выражение зависимости угла нейтральной линии от коэффициентов многочлена Лагранжа. Точность интерполяции в данном случае составила в среднем 11%.
Слайд 25
Разработан алгоритм работы устройства, который по значениям температуры и вычисленному положению нейтральной линии корректирует степень градиентного нагрева или мощность нагрева. Контроль происходит автоматически, в течение всего процесса гибки.
Слайд 26
На слайде показана конструкция разработанного устройства. Основными ее частями являются: блок компенсации системы нагрева, включающий трансформатор и блок конденсаторов, модуль линейного перемещения, перемещающий блок компенсации индуктор, закрепленный на блоке компенсации. В кольцо индуктора встроены 4 датчика температуры.
Слайд 27
Вкратце озвучу основные выводы работы.
- основным дефектом при гибке с узкозональным нагревом является утонение стенки, а наилучшим методом брьбы с утонением для тонкостенных труб – градиентный нагрев;
- разработанная математическая модель показала, что при гибке поворотным рычагом с постоянной температурой нагрева нейтральная линии совпадает со средней, а уменьшение утонения в результате применения градиентного нагрева не зависит от радиуса гиба и линейно зависит от степени градиентного нагрева;
- разработана конечно-элементная модель, подтвердившая выводы аналитической модели;
- проведены эксперименты также подтвердили выводы аналитической модели. Получено снижение утонения на 6% при гибке труб 36х2 мм из стали ВНС-16, что позволяет снизить массу трубопровода на 8%;
- разработано устройство для создания и контроля градиентного нагрева в автоматизированном режиме, а также алгоритм его работы.
Слайд 28
Таким образом научной новизной работы является математическая модель образования утонения при гибке тонкостенных труб с узкозональным нагревом, учитывающая смещение нейтральной линии, а также аналитические зависимости, полученные на ее основе. Кроме того, научной новизной обладает предложенная конечно-элементная модель образования утонения.
Практическую значимость составляет разработанные станок для гибки труб СГИН-120, а также конструкция и алгоритм работы устройства для создания и контроля градиентного нагрева.
Слайд 29
На этом доклад окончен, спасибо за внимание.

По докладу, автореферату и диссертации были заданы следующие вопросы:
Д.т.н., профессор Власов А.В.:
Вопрос: На 10 плакате приведены допущения конечно-элементой модели. Как вы оцениваете, какая ошибка вносится, насколько адекватность таких допущений справедлива? Для примера деформируется металл в зоне нагрева, это означает что никакой переходной зоны нет, хотя можно было бы смоделировать с учетом переходной зоны. Второе допущение – граничное условие «плоскость» двигается как плоскость. Наверное, это справедливо для начального момента? 
Ответ: Допущения были сделаны, исходя из предположения, что утонение не зависит от угла гиба, что было подтверждено экспериментально. Основную погрешность вносит допущение об отсутствии овализации. В данной модели учесть овализацию практически невозможно, потому что инструменты не моделируются. По результатам экспериментов было получено, что конечно-элементная модель дает завышенные результаты, поскольку поперечная сила присутствует, а овализация, которую вызывает данная сила, отсутствует.
Вопрос: Можно ли было учесть овализацию? Как можно было бы изменить модель?
Ответ: Для этого необходимо значительно усложнить модель. Решение не учитывать овализацию принято осознанно, чтобы упростить модель и повысить точность расчета деформаций стенки. Усложнение модели снижает точность расчетов. Такие исследования существуют, однако точность определения утонения в этом случае очень низкая. Для получения более точных результатов необходимы дальнейшие исследования.
Вопрос: Еще один вопрос касается этого же плаката. Применяется упруго-пластическая модель материала с линейным упрочнением. Нигде на плакатах не отражено, откуда взято это свойство упрочнения, откуда взят модуль упрочнения? Это ваши или сторонние эксперименты?
Ответ: Это справочные данные для выбранной стали.
Вопрос: Насколько эта сталь подходит под характеристику линейного упрочнения? Ansys вполне мог хорошо описать действительную кривую упрочнения. Или там упрочнения почти нет?
Ответ: Да, упрочнение практически отсутствует. Конечно, было бы лучше использовать более подробную характеристику, но для высоких температур данные очень скудны, в связи со сложностью проведения экспериментов. Кроме того, в работе исследовался не конкретный материал, а общие закономерности.


Д.т.н., доцент Лавриненко В.Ю.:
Вопрос: Скажите пожалуйста, возвращаясь к этому слайду, из каких соображений была выбрана высота рассматриваемого элемента?
Ответ: Высота элемента соответствует ширине зоны нагрева. Она выбиралась равной двум толщинам стенки, то есть 4 мм.
Вопрос: Еще один вопрос. Вы предлагаете новую конструкцию машины – там, где регулируется зазор с индуктором (слайд 23). Она достаточно трудоемка в изготовлении, имеет сложную систему управления. Может быть, экономически целесообразно использовать водяное охлаждение для создания градиентного нагрева, как на существующей машине? Еще один вопрос. Есть ли зарубежные аналоги машин с данным принципом работы?
Ответ: Метод создания градиентного нагрева водяным охлаждением не очень хорошо себя показал, поскольку процесс гибки нестабилен. В предложенном способе создания градиентного нагрева контроль осуществляется автоматически в течение всего процесса гибки – система ЧПУ подстраивает положение индуктора. Это не столь трудоемко, можно обойтись модернизацией существующего оборудования. По вопросу наличия аналогов. Градиентный нагрев применяется на практике – обычно устанавливают кольцо индуктора в одном положении в течение всего процесса гибки, на основании эмпирических наблюдений.
Вопрос: То есть положение не регулируется?
Ответ: Да, положение индуктора не регулируется.

Д.т.н., профессор Семенов М.Ю.:
Вопрос: Подчеркните те уравнения, которые вы получили аналитически.
Ответ: (Слайд 6). Здесь рассмотрена гибка при постоянной температуре нагрева. Получено выражение для определения утонения при гибке поворотным рычагом и утонение при гибке чистым моментом. Гибка чистым моментом не применяется, поэтому данное выражение представляет чисто теоретический интерес. (Слайд 8). Для гибки с градиентным нагревом получены выражения для определения утонения при линейном распределении напряжений текучести и при ступенчатом распределении. (Слайд 9). Также было получено общее выражение, определяющее общую закономерность применения градиентного нагрева. Это зависимость относительного уменьшения утонения от степени градиентного нагрева. 
Вопрос: То есть фактически вы решали интегральные уравнения?
Ответ: Да решались интегральные уравнения.
Вопрос: И еще вопрос. Каким образом реализован разработанный алгоритм работы устройства для создания градиентного нагрева.
Ответ: Он не реализован, только разработан для предлагаемого устройства.
Вопрос: Я правильно понял, что численные расчеты выполнялись в Ansys?
Ответ: Конечно-элементное моделирование выполнялось в Ansys, а остальные численные расчеты в Mathcad.
Вопросов к соискателю больше нет.

Председатель совета д.т.н., профессор Колесников А.Г. предоставляет слово научному руководителю, д.т.н., профессору Евсюкову С.А. (Отзыв прилагается).
Вопросов к научному руководителю нет.

Ученый секретарь к.т.н., доцент Плохих А.И. зачитывает заключение ФГБОУ ВО \thesisOrganizationShort\, где была выполнена работа. (Заключение прилагается).

Ученый секретарь к.т.н., доцент Плохих А.И. зачитывает отзыв ведущей организации – ФГБОУ ВО «Тульский государственный университет». (Отзыв прилагается. Отзыв положительный, имеются отдельные замечания).

Ученый секретарь к.т.н., доцент Плохих А.И. сообщает, что на автореферат получено 8 отзывов из следующих организаций:
ФГБОУ ВО «СибАДИ», подписан д.т.н., проф. кафедры эксплуатации и сервиса транспортно-технологических машин и комплексов в строительстве Кузнецовой В.Н. Отзыв положительный с замечаниями:
1) Осталось неясным, отличаются ли допущения и граничные условия, принятые автором диссертации для расчета утонения стенки трубы при гибке с поворотным рычагом, проведенные с использованием «инженерного» метода, и для моделирования процесса гибки трубы в САПР «ANSYS» в среде «Workbench».
2) Не указано, в чем заключается теоретическая значимость выполненных автором работы исследований.
«ВМЗ» – филиала АО «ГКНПЦ им. М.В. Хруничева», подписан заместителем директора завода – главным инженером Кольцовым В.И. и к.т.н., главным технологом Юхневичем С.С. Отзыв положительный с замечанием:
Как недостаток необходимо отметить недостаточность освещения влияния уменьшения утонения и применения градиентного нагрева на эксплуатационные свойства получаемых трубопроводов и отсутствие информации о режимах нагрева труб из различных материалов.
НИТУ «МИСиС», подписан д.т.н., проф. кафедры обработки металлов давлением Галкиным С.П. Отзыв положительный с замечаниями:
1) Из содержания автореферата не представляется возможным установить происхождение уравнения равновесия, представленных на стр. 6.
2) В автореферате не указано в каком интервале радиусов гиба справедливо приведенное на стр. 7 утверждение о том, что нейтральная линия при гибке совпадает со средней линией трубы.
ФГБОУ ВО «БГТУ», подписан к.т.н., доц., заместителем первого проректора по учебной работе Василенко Ю.В. Отзыв положительный с замечанием:
На фотографиях образцов изогнутых труб, приведенных в диссертационной работе, видно заметное уменьшение высоты поперечных сечений, данные ее измерения не приведены. Если оно не проводилось, то это явное упущение. Хотя в работе соискателя центральное место занимает утонение стенки трубы, но игнорировать овальность сечений при этом не следовало.
НИУ «БелГУ», подписан д.т.н., проф. кафедры материаловедения и нанотехнологий Афониным А.Н. Отзыв положительный с замечаниями:
1) В качестве последнего пункта научной новизны работы следует признать не саму конечно-элементную модель процесса гибки, а выявленные с ее помощью закономерности.
2) Из работы не ясно, какая модель реологических свойств материала заготовки была принята при конечно-элементном моделировании.
ООО «ИЛМиТ», подписан к.т.н., руководителем проекта департамента деформируемых сплавов и композиционных материалов Легких А.Н. и д.т.н., генеральным директором Дьяченко А.Н. Отзыв положительный с замечанием:
В математической модели материала принимается жестко-пластическая модель материала. Желательно оценить влияние упрочнения металла на процесс образования утонения.
ООО «КванторФорм», подписан, к.т.н., инженером отдела технической поддержки Харсеевым В.Е. и к.т.н., генеральным директором Стебуновым С.А. Отзыв положительный с замечаниями:
1) Разработанная модель гибки в комплексе Ansys рассматривает деформирование только зоны нагрева, что позволяет повысить скорость расчетов, однако не позволяет учесть влияние овализации и угла гибки на утонение.
2) В тексте автореферата не приведено обоснование выбора температуры нагрева при моделировании методом конечных элементов.
Самарского университета, кафедры обработки металлов давлением, подписан д.т.н., проф. Поповым И.П. и академиком РАН, д.т.н., проф., зав. кафедрой 
Гречниковым Ф.В. Отзыв положительный с замечанием:
В автореферате диссертационной работы отсутствуют графические зависимости, иллюстрирующие результаты проведенных исследований, что несколько затрудняет их восприятие и анализ.

Председатель совета д.т.н., профессор Колесников А.Г. предоставляет слово соискателю для ответа на замечания отзыва ведущей организации и отзывов на автореферат.

Долгополов М.И. отвечает на замечания в отзыве ведущей организации и отзывах на автореферат:
Отзыв ведущей организации:
1) С замечанием согласен. Эксперименты проводились для труб, реально применяющихся в ЖРД. При этом положения теории должны быть верны и для труб из других материалов. Эффективность градиентного нагрева при этом будет определяться зависимостью напряжений текучести от температуры и допустимыми границами температуры нагрева.
2) Границы применения определяются допущениями, принятыми при разработке математической модели, особенно допущением об отсутствии овализации поперечного сечения. На практике овализация становится слишком большой при радиусе гиба меньше 1,5D.
3) Применение градиентного нагрева возможно только в интервале допустимых температур гибки, определенного для каждого материала. Таким образом, изменение микроструктуры материала при применении градиентного нагрева не отличается от простой гибки с индукционным нагревом.
4) Данный график относится к случаю гибки «чистым» моментом и представляет исключительно теоретический интерес, так как данная схема гибки на практике не применяется. Аналогичные исследования других авторов мне неизвестны.
5) Главным критерием выбора функций являлась их монотонность на всем интервале, так как исследовался способ создания градиентного нагрева смещением кольца индуктора, при котором зазор между индуктором и трубой, а значит и температура нагрева, изменяется также монотонно.
6) С замечанием согласен. Для разработки более общей теории необходимы дальнейшие исследования.
7) С замечанием согласен. Отказ от моделирования овализации позволил значительно упростить модель и увеличить точности расчетов изменения деформации стенки. Для создания более сложной модели необходимы дальнейшие исследования.
8) С замечанием согласен. Исходя из производственного опыта можно сказать, что овализация и вероятность гофрообразования уменьшается с уменьшением ширины зоны нагрева, утонение же стенки не зависит от ширины зоны нагрева.

Отзыв СибАДИ:
1) При моделировании учитывалось упрочнение материала, а также моделировалось действие поворотного рычага, включая действие поперечной силы.
2) Наибольшую теоретическую значимость имеет определенная закономерность смещения нейтральной линии при гибке тонкостенных труб с узкозональным нагревом.


Отзыв «ВМЗ» – филиала АО «ГКНПЦ им. М.В. Хруничева»:
С замечанием согласен. Уменьшение утонения в результате применения градиентного нагрева позволяет увеличить надежность и прочность существующих трубопроводов.
Отзыв НИТУ «МИСиС»:
1) Данное уравнение получается, если в исходных общеизвестных уравнениях равновесия для цилиндрической системы координат сократить все слагаемые, соответствующие нулевым компонентам тензора напряжений (все компоненты, за исключением, осевых и круговых нормальных напряжений).
2) Данное утверждение справедливо, когда действуют допущения, принятые при разработке математической модели – при радиусе гиба меньше 1,5D.
Отзыв ФГБОУ ВО «БГТУ»:
С замечанием согласен.
Отзыв НИУ «БелГУ»:
1) С замечанием согласен.
2) В моделировании использовалась упруго-пластическая модель материала с линейным упрочнением.
Отзыв ООО «ИЛМиТ»:
Оценка влияния упрочнения сделана при конечно-элементном расчете. Упрочнение значительно влияет на значения напряжений, но практически не влияет на утонение. Упрочнение также может быть учтено и в аналитической модели, однако это усложнит расчеты и не позволит получить аналитическое решение.
Отзыв ООО «КванторФорм»:
1) С замечанием согласен. Для создания более сложной модели необходимы дальнейшие исследования.
2) Выбор температур нагрева был продиктован наличием подробных справочных данных о механических свойствах материала при данных температурах.
Отзыв Самарского университета:
С замечанием согласен.

Председатель совета д.т.н., профессор Колесников А.Г. предоставляет слово официальному оппоненту д.т.н., профессору Вдовину Сергею Ивановичу. (Отзыв прилагается. Отзыв положительный, имеются отдельные замечания).
Вопросов к официальному оппоненту нет.

Долгополов М.И.: отвечает на замечания официального оппонента д.т.н., профессора \opponentOneWhomFioShort:
1) ответ 1
2) ответ 2
3) ответ 3.

Председатель совета д.т.н., профессор Колесников А.Г. предоставляет слово официальному оппоненту к.т.н., профессору кафедры обработки материалов давлением и аддитивных технологий ФГБОУ ВО «Московский политехнический университет» Шпунькину Николаю Фомичу. (Отзыв прилагается. Отзыв положительный, имеются отдельные замечания).
Вопросов к официальному оппоненту нет.

Долгополов М.И. отвечает на замечания официального оппонента к.т.н., профессора Шпунькина Н.Ф.:
1) Зависимость разницы температур, а значит и утонения, от смещения индуктора очень сложна и не являлась предметом рассмотрения данной диссертации. В данной работе я исходил из условия, что данная зависимость неизвестна, поэтому была предложена конструкция устройства, измеряющего фактическое распределение температуры при гибке и корректирующего, в зависимости от измерений, положение индуктора.
2) Эффективность применения градиентного нагрева зависит от зависимости напряжений текучести материала трубы от температуры. Чем более «крутая» данная зависимость в диапазоне температур гибки, тем эффективнее будет применение градиентного нагрева.
3) При изготовлении трубопроводов ракетно-космической техники градиентный нагрев может применяться для снижения их массы, благодаря уменьшению предельного радиуса гиба. Кроме того, применение градиентного нагрева может увеличить прочность существующих трубопроводов, так как изгиб трубы является, как правило, наиболее слабым местом при воздействии внутреннего давления.


В последующей дискуссии выступили:
Д.т.н., профессор Полянский В.М.: 
Один из принципиальных моментов для трубопроводов – утонение. Еще один принципиальный момент – овализация. Она в работе принята отсутствующей. На самом деле это очень актуальный вопрос для использования трубопроводов. В целом оценка работы положительная.

Д.т.н., профессор Власов А.В.: 
Сразу скажу работа мне очень понравилась. Наш сегодняшний соискатель использовал аналитические методы, предложил численное решение, способен спроектировать машину, способен создать технологию. Это комплексный специалист. Таких нам нужно побольше. Сразу говорю, что я хочу проголосовать «за». Ну а теперь несколько критических замечаний. Мне нравится большое количество выкладок – это показывает скрупулезность соискателя. Он показывает: вот смотрите, проверяйте, все что я сделал – здесь написано. Хотите проверяйте, хотите вместо меня считайте. Такая открытость мне всегда прельщает, потому что иногда кладут в диссертации, как в автореферате уравнение и конечный результат. И либо ты за него попытайся все решить, но, извините, на это времени нет, либо доверяй человеку, что он сделал сам. Здесь есть возможность проверить полностью все решение. Ну и теперь по поводу численного решения. На мой взгляд Михаил Игоревич более сложные модели создавал, когда учился в институте, в курсе, в котором мы его обучали. Поэтому, наверное, в диссертации можно было сделать более сложную модель и учесть влияние соседних слоев, переходных температур. Чем хороша аналитическая модель – ты получаешь формулу. Для этого ты делаешь большое количество допущений. Причем эти допущения надо сделать так, чтобы «не выплеснуть с водой ребенка», чтобы осталась адекватность модели. В численном решении хотелось бы этих допущений иметь поменьше. Потому что тезис, как отвечал наш сегодняшний соискатель, что он сделал столько допущений, чтобы увеличить точность численного решения. Вот с таким подходом я абсолютно не согласен, но в целом, все равно я буду голосовать «за». Спасибо большое.

Д.т.н., профессор Семенов М.Ю.: 
Что мне хотелось бы сказать по поводу работы, представленной сегодня. На меня больше всего произвело впечатление, что соискатель владеет таким тонким аппаратом, как решение интегральных уравнений. Это бывает очень нечасто в наших работах. И даже, конечно, определенные известные недостатки Ansys и его ограничения уходят в тень, потому что модель, выполненная в Ansys, является не основной, а вспомогательной, средством дополнительной проверки адекватности полученных результатов. Мне кажется внимание, уделенное достоверности исследований, положительно характеризует данную работу. На мой взгляд данная работа заслуживает высокой оценки, а соискатель заслуживает присуждения ученой степени кандидата технических наук. Спасибо.

Председатель совета, д.т.н., профессор Колесников: Я подведу итог. Наша специальность называется технологии и машины обработки давлением. Эта работа демонстрирует все то, что отражено в названии. Здесь очень хорошо сочетаются исследования в области технологий и завершаются машиной, которая работает на сегодняшний день и действует. 
Желающих выступить больше нет.

Председатель совета д.т.н., профессор Колесников А.Г. предоставляет заключительное слово соискателю.
Долгополову М.И. в заключительном слове соискатель выражает благодарность научному руководителю, коллективу кафедры технологий обработки давлением \thesisOrganizationShort\, коллективу КБ НПО «Техномаш», официальным оппонентам и сотрудникам ведущей организации, а также членам диссертационного совета за конструктивную критику работы.

Избирается счетная комиссия в составе: д.т.н., доцент Курганова Ю.А., д.т.н., профессор Семенов М.Ю., д.т.н., старший научный сотрудник Колмаков А.Г.
Проводится тайное голосование.
Д.т.н., доцент Курганова Ю.А. (председатель счетной комиссии): объявляет результаты тайного голосования:
Состав диссертационного совета утвержден в количестве 24 человек.
Присутствовало на заседании 17 члена совета, в том числе докторов наук по профилю рассматриваемой диссертации – 8 человек.
Роздано бюллетеней – 17.
Осталось не розданных бюллетеней – 7.
Оказалось в урне бюллетеней – 17.
Результаты голосования по вопросу о присуждении ученой степени кандидата технических наук Долгополову Михаилу Игоревичу:
За – 17.
Против – нет.
Недействительных бюллетеней – нет.
Протокол счетной комиссии утверждается единогласно.

Председатель комиссии диссертационного совета д.т.н., профессор 
Власов А.В.: оглашает проект заключения по работе \thesisAuthorLastNameFrom.
После обсуждения и принятия поправок заключение принимается единогласно открытым голосованием.

\clearpage