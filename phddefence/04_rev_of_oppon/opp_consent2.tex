%==========================Шапка справа Сверху

\hfill\parbox{6cm}{
	\centerline{УТВЕРЖДАЮ}
	\centerline{\leadingOrganizationHeadPos\---}
	\leadingOrganizationTitle
	
	\ktn~\leadingOrganizationHead\
	\linebreak
	{\hbox to 6cm{\hrulefill}}
	{\hbox to 6cm{<<\rule{7mm}{0.4pt}>>\hrulefill~\number\year\,г.}}}
\vspace{0.5cm}


\section{Согласие 2}



105005, Москва, 2-я Бауманская ул., д.5.
Председателю диссертационного совета \defenseCouncilNumber\
д.т.н., профессору Колесникову А.Г.
от к.т.н., профессора Шпунькина Н.Ф.





Даю своё согласие на оппонирование диссертации \thesisAuthorLastNameFromFull\ на тему: <<\thesisTitle>>, представленной на соискание ученой степени кандидата технических наук по специальности \thesisSpecialtyNumber\ -– <<\thesisSpecialtyTitle>>.





Кандидат технических наук, профессор		Шпунькин Николай Фомич








Сведения об официальном оппоненте

Шпунькин Николай Фомич кандидат технических наук (05.02.09), профессор кафедры обработки материалов давлением и аддитивных технологий ФГБОУ ВО «Московский политехнический университет».

Список основных публикаций
1. S.A. Tipalin, Michael A. Petrov, N.F. Shpunkin To the Influence of the Deformation Speed on Hardening Process during the Cold Sheet Forming // Solid State Phenomena. 2018. V. 284, P. 513-518. DOI: 10.4028/www.scientific.net/SSP.284.513.
2. Шпунькин Н.Ф., Типалин С.А. Малогабаритная профилегибочная машина // Заготовительные производства в машиностроении. 2017. Т.15. №2. 
С. 62-66.
3. Шпунькин Н.Ф., Типалин С.А. Основы расчета параметров штамповки листовых деталей и оценка их технологичности: учебное пособие. М.: Ун-т машиностроения, 2016. 185 с.
4. Экспериментальное исследование осесимметричной формовки многослойного материала / Шпунькин Н.Ф. [и др.] // Известия Московского государственного технического университета МАМИ. 2015. Т.1, № 1(23). С.63-69. 
5. Шпунькин Н.Ф, Типалин С.А. Технологичность штампованных листовых деталей: учебное пособие. М.: Ун-т машиностроения, 2015. 72 с.
6. Шпунькин Н.Ф., Типалин С.А. Конструкция малогабаритной опытно-производственной профилегибочной машины // Известия Московского государственного технического университета МАМИ. 2015. Т.2, №1(23). С.57-63.
7. Бондарь В.С., Типалин С.А., Шпунькин Н.Ф. Изгиб и скручивание листа, М.: Ун-т машиностроения, 2014. 211 с.
8. Влияние изменения скорости деформации на характер упрочнения материала/ Шпунькин Н.Ф. [и др.] // Известия Московского государственного технического университета МАМИ. 2014. Т.2, №4(22). С.13-16.
9. Типалин С.А., Шпунькин Н.Ф., Сапрыкин Б.Ю. Определение свойств листового демпфирующего материала с упруговязким соединительным слоем при сдвиговой деформации // Известия Московского государственного технического университета МАМИ. 2014. Т.2, №2(20). С. 99-103.
10. Numerical and experimental investigation of deep drawing of sandwich panels / Shpunkin N.[ et al.] // Key engineering materials. 2014. V.611-612, P.1627-1636. DOI: 10.4028/www.scientific.net/KEM.611-612.1627.	



Официальный оппонент		Н.Ф. Шпунькин

\clearpage