
\section{Предполагаемые оппоненты и ведущая организация}


Соискатель: \thesisAuthorShort\ 
Тема: \thesisTitle.
\opponentOneFio\
доктор технических наук, профессор кафедры «Машиностроения» ФГБОУ ВО «ОГУ имени И.С. Тургенева» (бывший, в настоящее время не работает).
1. Вдовин С.И., Зайцев А.И., Татарченков Н.В. Математическое моделирование волнообразования при гибке труб // Фундаментальные и прикладные проблемы техники и технологии. 2018. № 1 (327). С. 62-65.
2. Вдовин С.И., Зайцев А.И. Расчет изгиба трубы с деформируемым сечением // Кузнечно-штамповочное производство. Обработка материалов давлением. 2017. № 5. С. 10-15.
3. Vdovin S.I., Lunin K.S. Pipe bending on the round cam with a longitudinal compression // Кузнечно-штамповочное производство. Обработка материалов давлением. 2016. № 1. С. 3-6.	
4. Экспериментальная установка для гибки труб наматыванием / С.И. Вдовин[ и др.] // Фундаментальные и прикладные проблемы техники и технологии. 2016. № 1 (315). С. 35-40.
5. Вдовин С.И., Зайцев А.И., Михайлов В.Н. Свободный изгиб трубы моментом // Фундаментальные и прикладные проблемы техники и технологии. 2016. № 5 (319). С. 77-79.
6. Вдовин С.И., Зайцев А.И., Михайлов В.Н. Стесненный изгиб трубы моментом // Фундаментальные и прикладные проблемы техники и технологии. 2016. № 6 (320). С. 82-86.
7. Вдовин С.И., Зайцев А.И., Лунин К.С. Стесненный изгиб трубы моментом // Фундаментальные основы механики. 2016. № 1. С. 98-101.
8. Вдовин С.И., Татарченков Н.В., Лунин К.С. Позиционирование дорна при гибке труб // Заготовительные производства в машиностроении. 2015. № 2. С. 26-28.
9. Инженерная теория гибки труб и изгиб моментом / С.И. Вдовин[ и др.] // Кузнечно-штамповочное производство. Обработка материалов давлением. 2015. 
№ 6. С. 3-6.	
10. Теоретическое исследование пластического изгиба трубы с деформируемым сечением / С.И. Вдовин[ и др.] // Фундаментальные и прикладные проблемы техники и технологии. 2015. № 5-2 (313). С. 262-265.
11. Вдовин С.И., Лунин К.С., Федоров Т.В. Аппроксимация перемещений в расчетах гибки труб // Фундаментальные и прикладные проблемы техники и технологии. 2015. № 6 (314). С. 104-107.	
12. К расчету пружинения при гибке труб / С.И. Вдовин[ и др.] // Фундаментальные и прикладные проблемы техники и технологии. 2014. № 2 (304). С. 77-81.

Шпунькин Николай Фомич
кандидат технических наук, профессор кафедры «Обработка материалов давлением и аддитивные технологии» ФГБОУ ВО «Московский политехнический университет»
1. S.A. Tipalin, Michael A. Petrov, N.F. Shpunkin To the Influence of the Deformation Speed on Hardening Process during the Cold Sheet Forming // Solid State Phenomena. 2018. V. 284, P. 513-518. DOI: 10.4028/www.scientific.net/SSP.284.513.
2. Шпунькин Н.Ф., Типалин С.А. Малогабаритная профилегибочная машина // Заготовительные производства в машиностроении. 2017. Т. 15. № 2. С. 62-66.
3. Шпунькин Н.Ф., Типалин С.А. Основы расчета параметров штамповки листовых деталей и оценка их технологичности: учебное пособие. М.: Ун-т машиностроения, 2016. - 185 с.
4. Экспериментальное исследование осесимметричной формовки многослойного материала / Шпунькин Н.Ф.[ и др.] // Известия Московского государственного технического университета МАМИ. 2015. Т. 1, № 1 (23). С. 63-69. 
5. Шпунькин Н.Ф, Типалин С.А. Технологичность штампованных листовых деталей: учебное пособие. М.: Ун-т машиностроения, 2015. 72 с.
6. Шпунькин Н.Ф., Типалин С.А. Конструкция малогабаритной опытно-производственной профилегибочной машины // Известия Московского государственного технического университета МАМИ. 2015. Т. 2, № 1 (23). С. 57-63.
7. Бондарь В.С., Типалин С.А., Шпунькин Н.Ф. Изгиб и скручивание листа, М.: Ун-т машиностроения, 2014. 211 с.
8. Влияние изменения скорости деформации на характер упрочнения материала/ Шпунькин Н.Ф.[ и др.] // Известия Московского государственного технического университета МАМИ. 2014. Т. 2, № 4 (22). С. 13-16.
9. Типалин С.А., Шпунькин Н.Ф., Сапрыкин Б.Ю. Определение свойств листового демпфирующего материала с упруговязким соединительным слоем при сдвиговой деформации // Известия Московского государственного технического университета МАМИ. 2014. Т. 2, № 2 (20). С. 99-103.
10. Numerical and experimental investigation of deep drawing of sandwich panels / Shpunkin N.[ et al.] // Key engineering materials. 2014. V. 611-612, P. 1627-1636. 
DOI: 10.4028/www.scientific.net/KEM.611-612.1627.	
ФГБОУ ВО «Тульский государственный университет»
1. Чудин В.Н., Черняев А.В., Булычев В.А. Изотермическая прошивка патрубков с наклонным фланцем // Заготовительные производства в машиностроении. 2019. Т. 17, № 3. С. 110-113.
2. Черняев А.В., Гладков В.А. Расчетная модель выдавливания внутреннего фланцевого утолщения на трубе // Заготовительные производства в машиностроении. 2018. Т. 16. № 3. С. 116-119.
3. Чудин В.Н., Пасынков А.А. Выдавливание элементов трубопроводов при вязкопластическом деформировании // Технология машиностроения. 2018. № 10. С. 20-24.
4. Черняев А.В., Гладков В.А., Чудин В.Н. Формообразование давлением кольцевых ребер на трубе // Технология машиностроения. 2018. № 9. С. 10-14.
5. Кухарь В.Д., Киреева А.Е., Сорвина О.В. Калибровка внутренней поверхности трубчатой заготовки давлением импульсного магнитного поля // Вестник машиностроения. 2017. № 9. С. 41-43.
6. Грязев М.В., Ларин С.Н., Пасынков А.А. Оценка влияния технологических факторов на предельный коэффициент раздачи трубных заготовок коническим пуансоном // Заготовительные производства в машиностроении. 2017. Т. 15. № 6. С. 255-260.
7. Грязев М.В., Трегубов В.И., Пасынков А.А. Влияние механических свойств на предельные возможности раздачи трубных заготовок коническим пуансоном // Кузнечно-штамповочное производство. Обработка материалов давлением. 2017. 
№ 10. С. 3-7.
8. Подход к оценке напряженно-деформированного состояния трубной заготовки при раздаче коническим пуансоном / Грязев М.В.[ и др.] // Известия Тульского государственного университета. Технические науки. 2017. № 1. С. 171-177.
9. Кухарь В.Д., Маленичев Е.С. Разработка математических моделей для анализа процессов магнитно-импульсной штамповки продольных выступов на трубчатых деталях // Известия Тульского государственного университета. Технические науки. 2017. № 1. С. 218-223.
10. Грязев М.В., Ларин С.Н., Пасынков А.А. Подход к оценке напряженно-деформированного состояния трубной заготовки при раздаче коническим пуансоном // Известия Тульского государственного университета. Технические науки. 2017. № 3. С. 3-9.
11. Грязев М.В., Ларин С.Н., Пасынков А.А. Оценка влияния механических свойств трубной заготовки на ее напряженное состояние и силу при раздаче // 
Известия Тульского государственного университета. Технические науки. 2017. 
№ 6. С. 3-10.
12. Пасынков А.А., Аккуратнова А.С. Оценка напряженно-деформированного состояния и возможностей формоизменения тонкостенной трубной заготовки из сплава вт14 при ее раздаче в изотермических условиях // Известия Тульского государственного университета. Технические науки. 2017. № 9-1. С. 286-291.
13. Грязев М.В., Ларин С.Н. Подход к разработке математической модели процесса раздачи трубы коническим пуансоном // Фундаментальные и прикладные проблемы техники и технологии. 2017. № 4-2 (324). С. 12-17.
14. Яковлев С.С., Трегубов В.И., Осипова Е.В. Предельные деформации при ротационной вытяжке с утонением стенки трубных заготовок из анизотропных материалов // Вестник машиностроения. 2016. № 3. С. 81-84.
15. Грязев М.В., Ларин С.Н., Пасынков А.А. Исследование напряженного и деформированного состояния и накопленных микроповреждений при обжиме трубной заготовки в конической матрице // Заготовительные производства в машиностроении. 2016. № 7. С. 13-17.

\clearpage
