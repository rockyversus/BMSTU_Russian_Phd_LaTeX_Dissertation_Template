
\section{Замечания}

===Глава 3
====Радиуса скруглений матриц====
-При исследовании радиусов скругления в главе 3 указано что исследованы комбинации где на 1м переходе вытяжки один радиус, а на втором переходе другой радиус - но это не описано и не отражено на графиках
- в радиусах скруглений описано влияние на складкообразование, но не описано влияние на толщину конечной детали
-Не описано какое влияние оказывает обрезка фланца и величина высоты борта при обжиме
- не описано влияние других технологических параметров на а) стабильность процесса б) толщину конечной детали (смазка, скорости движения пуансона)
- толщина стенки переведена в относительную, а параметры оснастки - не перведены

-однако, при диаметре d = 5 мм, происходит недоштамповки металла, что говорит о необходимости дальнейшей калибровки детали при помощи эластичного пуансона. А график притерия складкообразования ничего не показал - так как складки не появляется. Т.е. может получиться бракованная деталь, а критерий бдет иметь хорошие показатели

- на Осевая схема (эскиз) набираемого металла.не указан размер длины горизонтальной площадки и площадки которая идет под углом

- не описано как производилось разделение заготовок так как технология эта может существенно влиять на точность показателей

- не указано проводилась ли поверка средств измерения, измерения проводились поверенными редствами или нет ? 






Пояснения к замечаниям
1. Отзыв д.т.н., проф. \opponentOneWhomFioShort
%1.1 Касательные напряжения τρθ, присутствующие в п. 2.3.1 диссертации, не могут иметь место в принятой математической модели. Они появятся, если задать овальность сечений, приобретаемую при гибке, но и в этом случае будут отсутствовать в плоскости симметрии. 
Введение касательных напряжений в математической модели при рассмотрении гибки с градиентным нагревом необходимо, чтобы выполнялось условие равновесия сечения. При градиентном нагреве круговые напряжения в зонах растяжения и сжатия различны что, учитывая сохранение площади сечения стенки трубы при деформировании, должно вызывать появление крутящего момента и соответствующих касательных напряжений. Касательные напряжения в данной задаче, возможно, являются следствием несовершенства математической модели или принятых допущений.
%1.2 Не понятно, к какой точке сечения трубы относятся данные табл. 2. Если эта точка в плоскости симметрии (изгиба), то наличие в ней τρθ может говорить о некорректной постановке задачи КЭ-моделирования.
Таблица относится ко всему объему моделируемой трубы. В ней приведены диапазоны изменения значений компонентов тензора напряжений. Наличие касательных напряжений связано с тем, что при гибке поворотным рычагом на трубу действует также поперечная сила. Эта сила не учитывалась в аналитическом расчете – она вызывает овализацию, которая не рассматривалась. В моделировании овализация также не рассматривается (центральная часть стенки трубы закреплена от перемещений), но поперечная сила присутствует.
1.3 Объем диссертации завышен за счет мало значащих ссылок на источники и подробного описания математических преобразований, в то же время недостаточно раскрыта ее связь с проблематикой ракетно-космической отрасли.
С замечанием согласен. Основное преимущество гибки с градиентным нагревом для ракетно-космической отрасли – снижение массы трубопровода.

2. Отзыв к.т.н., проф. Шнунькина Н.Ф.
2.1 В работе описывается устройство, создающее градиентный нагрев смещением индуктора, однако недостаточны исследования данной технологии. Не исследована зависимость утонения стенки и распределения температуры от величины смещения индуктора.
Зависимость разницы температур, а значит и утонения, от смещения индуктора очень сложна и не являлась предметом рассмотрения данной диссертации. В данной работе я исходил из условия, что данная зависимость неизвестна, поэтому была предложена конструкция устройства, измеряющего фактическое распределение температуры при гибке и корректирующего, в зависимости от измерений, положение индуктора.
2.2 Следовало более подробно осветить вопрос о том, для каких труб применение градиентного нагрева является целесообразным.
Эффективность применения градиентного нагрева зависит от зависимости напряжений текучести материала трубы от температуры. Чем более «крутая» данная зависимость в диапазоне температур гибки, тем эффективнее будет применение градиентного нагрева.
2.3 Недостаточны исследования о влиянии градиентного нагрева на конкретные свойства трубопроводов ракетно-космической техники.
При изготовлении трубопроводов ракетно-космической техники градиентный нагрев может применяться для снижения их массы, благодаря уменьшению предельного радиуса гиба. Кроме того, применение градиентного нагрева может увеличить прочность существующих трубопроводов, так как зона изгиба трубы является, как правило, наиболее слабым местом при воздействии внутреннего давления.

3. Отзыв ведущей организации ТулГУ
3.1 В работе представлены результаты применения градиентного нагрева для труб из стали ВНС16, однако отсутствуют данные об эффективности применения градиентного нагрева с трубами из других материалов.
С замечанием согласен. Эксперименты проводились для труб, реально применяющихся в ЖРД. При этом положения теории должны быть верны и для труб из других материалов. Эффективность градиентного нагрева при этом будет определяться зависимостью напряжений текучести от температуры и допустимыми границами температуры нагрева.
3.2 Не указаны общие границы применения разработанной технологии, в зависимости от геометрии и свойств материала трубопровода.
Границы применения определятся допущениями, принятыми при разработке математической модели, особенно допущением об отсутствии овализации поперечного сечения. На практике овализация становится слишком большой при радиусе гиба меньше 1,5D.
3.3 Недостаточно освещен вопрос влияния градиентного нагрева на микроструктуру получаемых изделий.
Применение градиентного нагрева возможно только в интервале допустимых температур гибки, определенного для каждого материала. Таким образом изменение микроструктуры материала при применении градиентного нагрева не отличается от простой гибки с индукционным нагревом.
3.4 В главе 2 получен график зависимости угла нейтральной линии от относительного радиуса гиба (рис. 2.7). Однако отсутствует сравнение полученных результатов с исследованиями других авторов.
Данный график относится к случаю гибки «чистым» моментом и представляет исключительно теоретический интерес, так как данная схема гибки на практике не применяется. Аналогичных исследований других авторов мне не известно.
3.5 При рассмотрении процесса гибки с градиентным нагревом при произвольном распределении напряжений текучести для определения относительного утонения задается несколько видов функций (2.74), но обоснование выбора данных функций в диссертационной работе не приводится.
Главным критерием выбора функций являлась их монотонность на всем интервале, так как исследовался способ создания градиентного нагрева смещением кольца индуктора, при котором зазор между индуктором и трубой, а значит и температура нагрева, изменяется также монотонно.
3.6 Полученная математическая модель гибки труб с узкозональным индукционным нагревом, разработанная на базе «инженерного» метода, не дает возможности прогнозирования потери устойчивости стенки трубы на внутренней стороне гиба в виде гофрообразования.
С замечанием согласен. Для разработки более общей теории необходимы дальнейшие исследования.
3.7 В третьей главе с применением МКЭ выполнено моделирование процесса гибки с узкозональным индукционным нагревом без учета овализации поперечного сечения трубы. Однако не учет овализации поперечного сечения при моделировании ведет к завышению значений величины утонения стенки трубы.
С замечанием согласен. Отказ от моделирования овализации позволил значительно упростить модель и увеличить точности расчетов изменения деформации стенки. Для создания более сложной модели необходимы дальнейшие исследования.
3.8 В диссертационной работе было бы полезно провести исследования влияния ширины зоны нагрева на напряженно-деформированное состояние изгибаемой трубы, положение нейтральной линии, величину утонения, вероятность появления дефектов в виде овализации поперечного сечения и гофрообразования.
С замечанием согласен. Исходя из производственного опыта можно сказать, что овализация и вероятность гофрообразования уменьшается с уменьшением ширины зоны нагрева, утонение же стенки не зависит от ширины зоны нагрева.


4. Отзыв СибАДИ (д.т.н., проф. Кузнецова В.Н.)
4.1 Осталось неясным, отличаются ли допущения и граничные условия, принятые автором диссертации для расчета утонения стенки трубы при гибке с поворотным рычагом, проведенные с использованием «инженерного» метода, и для моделирования процесса гибки трубы в САПР «ANSYS» в среде «Workbench».
При моделировании учитывалось упрочнение материала, а также моделировалось действие поворотного рычага, включая действие поперечной силы.
4.2 Не указано, в чем заключается теоретическая значимость выполненных автором работы исследований.
Наибольшую теоретическую значимость имеет определенная закономерность смещения нейтральной линии при гибке тонкостенных труб с узкозональным нагревом.

5. Отзыв ВМЗ (к.т.н. Юхневич С.С.)
Как недостаток необходимо отметить недостаточность освещения влияния уменьшения утонения и применения градиентного нагрева на эксплуатационные свойства получаемых трубопроводов и отсутствие информации о режимах нагрева труб из различных материалов.
С замечанием согласен. Уменьшение утонения в результате применения градиентного нагрева позволяет увеличить надежность и прочность существующих трубопроводов.

6. Отзыв МИСиС (д.т.н., проф. Галкин С.П.)
6.1 Из содержания автореферата не представляется возможным установить происхождение уравнения равновесия, представленных на стр. 6.
Данное уравнение получается, если в исходных общеизвестных уравнениях равновесия для цилиндрической системы координат сократить все слагаемые, соответствующие нулевым компонентам тензора напряжений (все компоненты, за исключением, осевых и круговых нормальных напряжений).
6.2 В автореферате не указано в каком интервале радиусов гиба справедливо приведенное на стр. 7 утверждение о том, что нейтральная линия при гибке совпадает со средней линией трубы.
Данное утверждение справедливо, когда действуют допущения, принятые при разработке математической модели – при радиусе гиба меньше 1,5D.




7. Отзыв БГТУ (к.т.н., доц. Василенко Ю.В.)
В качестве замечания по работе следует отметить следующее. На фотографиях образцов изогнутых труб, приведенных в диссертационной работе, видно заметное уменьшение высоты поперечных сечений, данные ее измерения не приведены. Если оно не проводилось, то это явное упущение. Хотя в работе соискателя центральное место занимает утонение стенки трубы, но игнорировать овальность сечений при этом не следовало.
С замечанием согласен.

8. Отзыв БелГУ (д.т.н., проф. Афонин А.Н.)
8.1 В качестве последнего пункта научной новизны работы следует признать не саму конечноэлементную модель процесса гибки, а выявленные с ее помощью закономерности.
С замечанием согласен.
8.2 Из работы не ясно, какая модель реологических свойств материала заготовки была принята при конечноэлементом моделировании.
В моделировании использовалась упруго-пластическая модель материала с линейным упрочнением.

9. Отзыв ИЛМиТ РУСАЛ (к.т.н. Легких А.Н.)
К выполненной работе имеется замечание: в математической модели ма-териала принимается жестко-пластическая модель материала. Желательно оце-нить влияние упрочнения металла на процесс образования утонения.
Оценка влияния упрочнения сделана при конечно-элементном расчете. Упрочнение значительно влияет на значения напряжений, но практически не влияет на утонение. Упрочнение также может быть учтено и в аналитической модели, однако это значительно усложнит расчеты и не позволит получить аналитическое решение.

10. Отзыв КванторФорм (к.т.н. Харсеев В.А.)
10.1	Разработанная модель гибки в комплексе Ansys рассматривает деформирование только зоны нагрева, что позволяет повысить скорость расчетов, однако не позволяет учесть влияние овализации и угла гибки на утонение.
С замечанием согласен. Для создания более сложной модели необходимы дальнейшие исследования.
10.2	В тексте автореферата не приведено обоснование выбора температуры нагрева при моделировании методом конечных элементов.
Выбор температур нагрева был продиктован наличием подробных справочных данных о механических свойствах материала при данных температурах.

\clearpage