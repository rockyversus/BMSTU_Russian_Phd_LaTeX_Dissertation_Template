%тест 23 03 2024\documentclass[a4paper,12pt]{extarticle} %,twoside
%тест 23 03 2024%%%=================================
%тест 23 03 2024\input{Common/_packages}  % Пакеты общие для диссертации и автореферата
%тест 23 03 2024\input{Avtoreferat/_tablicy}               % Упрощённые настройки шаблона 
%тест 23 03 2024\input{Common/00_data}      % Основные сведения
%тест 23 03 2024\input{Common/_common_style}    % Стили общие для диссертации и автореферата
%тест 23 03 2024\input{Avtoreferat/_style}           % Стиль для автореферата
%тест 23 03 2024\input{biblio/biblatex}    % Реализация пакетом biblatex через движок biber
%тест 23 03 2024\DefineBibliographyStrings{russian}{number={\textnumero}}
%тест 23 03 2024\input{Common/03_nachalo} 
%тест 23 03 2024\input{Common/02_frazi} 
%тест 23 03 2024%%%======Титульный лист==============
%\begin{document}




%\chapter{Заключение диссертационного совета} \label{zak_dis_sovet}
%\addcontentsline{toc}{chapter}{Заключение диссертационного совета}    % Добавляем его в оглавление

%==========================Заголовок


\section{Заключение диссертационного совета \defenseCouncilNumber\ по диссертации \thesisAuthorLastNameFrom\ на соискание ученой степени кандидата наук}


%%%%%%%%%%%%%%%%%%%%%%%%%%%%Тело документа
%=========================Новизна

\textbf{Диссертационный совет отмечает, что на основании выполненных соискателем исследований:}

%\textit{разработана} новая научно-обоснованная - ТЕСТ ССЫЛОК {\zadachiONE} %{\noviznaTWO};

%\textit{предложены} новые {\noviznaFOUR};

%==========================Теоретическая Значимость
%\textbf{Теоретическая значимость} исследования обоснована тем, что:

%\textit{раскрыты} новые  новые {\noviznaFIVE};

%\textit{изучены} {\noviznaTHREE}.

%==========================Практическая Значимость
%\textbf{Значение полученных соискателем результатов исследования для практики подтверждается тем, что закреплены:}

%\textit{созданные} {\znachimostTWO} \textit{использованы} в {\leadingOrganizationTitle} при изготовлении {\chego};

%\textit{разработанная} {\znachimostSIX};

%\textit{определены} {\znachimostFOUR};

%{\znachimostFIVE}.

%==========================Достоверность
%\textbf{Оценка достоверности результатов исследования выявила:}

%\textit{для экспериментальных работ} {\dostovernostONE};

%\textit{разработанная} {\dostovernostTHREE};

%\textit{полученные} {\dostovernostTWO}.

%===========================Личный вклад
%\textbf{Личный вклад соискателя состоит в:}

%\textit{непосредственном} {\vkladONE}

%\textit{лично соискателем} {\vkladTWO}

%============================Вывод
Диссертационным советом сделан вывод о том, что диссертация является научно-квалификационной работой, в которой изложены научно-обоснованные технологические решения позволяющие провести, \MakeLowercase{{\thesisTitle}}, имеющие существенное значение для машиностроения (п. 9 Положения о присуждении ученых степеней).

\vspace*{4.5em plus .6em minus .5em}

\begin{center}
\begin{tabular}[c]{c m{4cm} l}
	
	     Председатель       &  												\\
	диссертационного совета &  			 & Колесников Александр				\\
	   д.т.н., профессор    & \hrulefill & Григорьевич 						\\			
	                        &            &                    			    \\
	   Ученый секретарь     &  												\\
	диссертационного совета &  												\\
	    к.т.н., доцент      & \hrulefill & Семенов Вячеслав Иванович
	    
\end{tabular}
\end{center}


%\end{document}
\clearpage