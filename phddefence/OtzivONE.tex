%тест 23 03 2024\documentclass[a4paper,14pt]{extarticle} %,twoside
%тест 23 03 2024%%%=================================
%тест 23 03 2024\input{Common/_packages}  % Пакеты общие для диссертации и автореферата
%тест 23 03 2024\input{Avtoreferat/_tablicy}               % Упрощённые настройки шаблона 
%тест 23 03 2024\input{Common/00_data}      % Основные сведения
%тест 23 03 2024\input{Common/_common_style}    % Стили общие для диссертации и автореферата
%тест 23 03 2024\input{Avtoreferat/_style}           % Стиль для автореферата
%тест 23 03 2024\input{biblio/biblatex}    % Реализация пакетом biblatex через движок biber
%тест 23 03 2024\DefineBibliographyStrings{russian}{number={\textnumero}}
%тест 23 03 2024\input{Common/03_nachalo} 
%тест 23 03 2024\input{Common/02_frazi}
%тест 23 03 2024\usepackage{mfirstuc}
%тест 23 03 2024%%%======Титульный лист==============
\begin{document}


\hfill\parbox{6cm}{
\centerline{В диссертационный совет}
\centerline{Д 212.141.04}
МГТУ им. Н.~Э.~Баумана
105005,~г.~Москва,~2-ая~Бауманская~ул.,~д.~5~стр.1
}
\vspace{1cm}

\begin{center}
    \section*{Отзыв Официального оппонента}
\end{center}


\subsubsection*{д.т.н., профессора Сосенушкина Евгения Николаевича
на диссертацию Ступникова Вадима Владимировича {\thesisTitle}, представленную на соискание ученой степени кандидата технических наук по специальности 05.02.09---Технологии и машины обработки давлением}

\textsc{Актуальность работы}: {\actualityTEXT}

\textsc{Содержание работы}. 
Диссертация состоит из введения, четырех глав основного текста, общих выводов. Общий объем диссертации составляет 160 страниц. Диссертация содержит 156 рисунков, 19 таблиц и список литературы из 164 наименований.

По своему содержанию, объему и оформлению диссертационная работа отвечает требованиям, предъявляемым ВАК РФ к кандидатским диссертациям.

Во введении обоснована актуальность темы, приведены цель и задачи исследования и сформулированы основные положения, выносимые автором на защиту.
Первая глава посвящена обзору литературных источников по исследуемому вопросу. 

\ldots

текст будет добавлен после завершения и утверждения текст автореферата

\ldots

На основании анализа технической литературы сформулированы цель и задачи работы.

\textsc{Во второй главе}

\ldots

текст будет добавлен после завершения и утверждения текст автореферата

\ldots

Определены основные направления экспериментальных исследований, требуемых для использования разработанной модели.

\textsc{Третья глава} посвящена

\ldots

текст будет добавлен после завершения и утверждения текст автореферата

\ldots


\textsc{В четвертой главе} приведена

\ldots

В заключении сформулированы основные выводы по работе.
В тексте приведено большое количество ссылок на использованные литературные источники. Некорректных заимствований в диссертации не обнаружено

\textsc{Научную новизну} имеют следующие результаты, полученные лично автором:

\begin{enumerate}
  \item {\noviznaONE};
  \item {\noviznaTWO};
  \item {\noviznaTHREE};
  \item {\noviznaFOUR};
  \item {\noviznaFIVE};
  \item {\noviznaSIX}.
\end{enumerate}

\textsc{Практическая значимость} работы заключается в:

% Значимость
\begin{enumerate}
    \item {\znachimostONE};
    \item {\znachimostTWO};
    \item {\znachimostTHREE};
    \item {\znachimostFOUR};
    \item {\znachimostFIVE};
    \item {\znachimostSIX};
    \item {\znachimostSEVEN}.
\end{enumerate}

\textsc{Степень обоснованности научных положений, выводов и рекомендаций, сформулированных в диссертации, и их достоверность подтверждается} корректностью постановки задач, использованием современных методов исследований, корректной формулировкой математической модели процесса, с учетом его характерных особенностей в программных комплексах AutoForm и LS-DYNA, обстоятельным подходом к выполнению экспериментальных исследований и, как следствие, удовлетворительным совпадением результатов расчетов и экспериментов. 

\textbf{Достоверность вывода \textnumero2} основана на результатах проведенных автором экспериментальных исследованиях по

\ldots

\textbf{Согласно выводу \textnumero3} математическая модель позволяет достоверно установить 

\ldots

\textbf{Вывод  \textnumero4} посвященный результатам верификации разработанной математической модели, обосновывает полученные расхождения в расчетах технологических параметров исследуемого процесса, разброс ошибки по различным параметрам лежит в диапазоне от 0,8 до 21,5\%.

\textbf{Вывод  \textnumero5} подтверждает одно из положений практической значимости диссертации о том, что разработанная методика позволила автору достоверно определить 

\ldots

\textsc{Замечания по выполненной работе}:

\begin{enumerate}
    \item Некоторые рисунки
    \item Глава \textnumero1 не содержит описания технологии раскатки роликами \ldots
    \item Уравнение \ldots странная запись с квадратными скобками
    \item В работе не проведен анализ границ применимости предложенной методики проектирования процесса 
    \item Не ясно, оказывает ли влияние исходная микроструктура на микроструктуру конечного изделия
    \item Не описано влияние предложенного технологического процесса на микроструктуру
    \item Как увязаны процессы дискретизации объема заготовки на конечные элементы с исходной микроструктурой металла
    \item Существует ли связь пошагового перестроения сетки конечных элементов модели с изменением структуры в процессе моделирования технологии обработки давлением
    \item Учтено ли в модели \ldots
    \item Отсутствует исследование, подтверждающее стабильность механических свойств
    \item В диссертации не показано влияние масштабного фактора на время выдержки
    \item Избыточный объем материалов по исследованию микроструктур, т.к. это область исследований смежной специальности 05.16.05 «Обработка металлов давлением» и они должны носить вспомогательный характер, например, использоваться в приложениях.
    \item \ldots
    \item \ldots
    
\end{enumerate}

\textsc{Соответствие диссертации паспорту специальности} 05.02.09 <<Технологии и машины обработки давлением>> наиболее значимыми являются исследования соответствующие области исследования научной специальности 05.02.09 <<Технологии и машины обработки давлением>>:
\ldots

\textsc{Заключение.}
Диссертация Ступникова Вадима Владимировича является самостоятельной и логически завершенной научно-квалификационной работой, в которой изложены новые научно-обоснованные технологические решения для обеспечения {\celTEXT} которые имеют существенное значение для развития машиностроения. Работа отвечает требованиям пункта 9 Положения ВАК РФ о порядке присуждения ученых степеней, предъявляемым к кандидатским диссертациям, а ее автор Ступников Вадим Владимирович заслуживает присуждения учёной степени кандидата технических наук по специальности 05.02.09---Технологии и машины обработки давлением.


\begin{center}
\begin{tabular}[c]{l m{3.1cm} r}

Официальный оппонент\\ д.~т.~н., профессор кафедры\\ систем пластического деформирования\\ ФГБОУ ВО «МГТУ «СТАНКИН»
 \\ <<Технологии обработки давлением>> & & Сосенушкин Е.Н. \\ 
  &   \\  

\end{tabular}
\end{center}

\begin{flushleft}
\begin{tabular}{l@{~}l@{~}l}
Адрес:  & 127055, г. Москва, Вадковский пер., 3-А \\
Телефон:    & +7(499)~972-95-27 \\
Email:    & sen@stankin.ru \\
\end{tabular}
\end{flushleft}

\end{document}