%тест 23 03 2024 \documentclass[a4paper,14pt]{extarticle} %,twoside
%тест 23 03 2024%%%=================================
%тест 23 03 2024\input{common/packages}  % Пакеты общие для диссертации и автореферата
%тест 23 03 2024\input{comment/_tablicy}               % Упрощённые настройки шаблона 
%тест 23 03 2024\input{Common/00_data}      % Основные сведения
%тест 23 03 2024\input{Common/_common_style}    % Стили общие для диссертации и автореферата
%тест 23 03 2024\input{Avtoreferat/_style}           % Стиль для автореферата
%тест 23 03 2024\input{biblio/biblatex}    % Реализация пакетом biblatex через движок biber
%тест 23 03 2024\DefineBibliographyStrings{russian}{number={\textnumero}}
%тест 23 03 2024\input{Common/03_nachalo} 
%тест 23 03 2024\input{Common/02_frazi} 
%тест 23 03 2024\usepackage{mfirstuc}
%тест 23 03 2024%%%======Титульный лист==============

%==========================Шапка справа Сверху

\hfill\parbox{6cm}{
	\centerline{УТВЕРЖДАЮ}
	\centerline{\leadingOrganizationHeadPos\---}
	\leadingOrganizationTitle
	
	\ktn~\leadingOrganizationHead\
	\linebreak
	{\hbox to 6cm{\hrulefill}}
	{\hbox to 6cm{<<\rule{7mm}{0.4pt}>>\hrulefill~\number\year\,г.}}}
\vspace{0.5cm}



\section{Акт об использовании результатов диссертационной работы}
%\vspace{30pt} %число перед fill = кратность относительно некоторого расстояния fill, кусками которого заполнены пустые места

Настоящий акт составлен в том, что результаты диссертационной работы инженера \thesisOrganizationShort\ \thesisAuthorShort\ переданы для дальнейшего использования в \leadingOrganizationTitle. В диссертационной работе \thesisAuthorShort\ предложил способ изготовления сферических деталей методом обжима цилиндрического стаканчика с предварительным набором металла, что позволяющем уменьшить расход металла и снизить разнотолщинность. Для этого использовалась разработанная методика анализа и прогнозирования появления дефектов в виде складок при обжиме. В результате были даны зависимости для определения геометрических размеров изделия при различных соотношениях толщины, диаметра и коэффициентов вытяжки и рекомендации по применению метода обжима с предварительным набором металла и процессов ему предшествующих.

Данные результаты представляю интерес для \leadingOrganizationTitle\ и будут учтены, в частности, при разработке технологических процессов изготовления цилиндрических и сферических деталей.
\vspace{0.5cm}
 
%\vspace{20pt} %число перед fill = кратность относительно некоторого расстояния fill, кусками которого заполнены пустые места
 
%\vspace*{4.5em plus .6em minus .5em}

\begin{center}
	\begin{tabular}[c]{c m{4cm} l}
		
		       Начальник         &            &                                      \\
		  технического отдела    &            &                                      \\
		          \ktn           & \hrulefill & Петров Петр Петрович                 \\
		                         &            &                                      \\
		        Инженер          &            &                                      \\
		\thesisOrganizationShort & \hrulefill & \thesisAuthor
	\end{tabular}
\end{center}

\clearpage
 
