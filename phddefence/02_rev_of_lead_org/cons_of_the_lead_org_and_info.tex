%==========================Шапка справа Сверху

\hfill\parbox{6cm}{
	\centerline{УТВЕРЖДАЮ}
	\centerline{\leadingOrganizationHeadPos\---}
	\leadingOrganizationTitle
	
	\ktn~\leadingOrganizationHead\
	\linebreak
	{\hbox to 6cm{\hrulefill}}
	{\hbox to 6cm{<<\rule{7mm}{0.4pt}>>\hrulefill~\number\year\,г.}}}
\vspace{0.5cm}



\section{согласие и СВЕДЕНИЯ О ВЕДУЩЕЙ ОРГАНИЗАЦИИ}



105005, Москва, 2-я Бауманская ул., д. 5.
Председателю диссертационного совета \defenseCouncilNumber\
ФГБОУ ВО \thesisOrganizationShort
д.т.н., профессору Колесникову А.Г.




Уважаемый Александр Григорьевич!

ФГБОУ ВО «Тульский государственный университет» (ТулГУ) дает согласие выступить в качестве ведущей организации и дать отзыв на диссертацию \thesisAuthorLastNameFromFull\ на тему: «Разработка технологии гибки тонкостенных трубопроводов с использованием узкозонального индукционного градиентного нагрева», представленной на соискание ученой степени кандидата технических наук по специальности \thesisSpecialtyNumber\ –- <<\thesisSpecialtyTitle>>.

Приложение: сведения о ведущей организации.





Проректор по научной работе ТулГУ, д.т.н., профессор		Воротилин М. С.








СВЕДЕНИЯ О ВЕДУЩЕЙ ОРГАНИЗАЦИИ
по диссертации \thesisAuthorLastNameFromFull\ на тему: <<\thesisTitle>> представленной на соискание ученой степени доктора технических наук по специальности 
\thesisSpecialtyNumber\ –- <<\thesisSpecialtyTitle>>.

Полное наименование	Федеральное государственное бюджетное образовательное учреждение высшего образования «Тульский государственный университет»
Сокращенное наименование	ФГБОУ ВО «Тульский государственный университет»
Почтовый адрес организации	300012, г. Тула, пр. Ленина, 92
Телефон	+7 (4872) 35-34-44
Адрес электронной почты	info@tsu.tula.ru
Веб-сайт	http://tsu.tula.ru

Список основных публикаций работников ведущей организации
1. Чудин В.Н., Черняев А.В., Булычев В.А. Изотермическая прошивка патрубков с наклонным фланцем // Заготовительные производства в машиностроении. 2019. Т. 17, № 3. С. 110-113.
2. Черняев А.В., Гладков В.А. Расчетная модель выдавливания внутреннего фланцевого утолщения на трубе // Заготовительные производства в машиностроении. 2018. Т. 16. № 3. С. 116-119.
3. Чудин В.Н., Пасынков А.А. Выдавливание элементов трубопроводов при вязкопластическом деформировании // Технология машиностроения. 2018. № 10. С. 20-24.
4. Черняев А.В., Гладков В.А., Чудин В.Н. Формообразование давлением кольцевых ребер на трубе // Технология машиностроения. 2018. № 9. С. 10-14.
5. Кухарь В.Д., Киреева А.Е., Сорвина О.В. Калибровка внутренней поверхности трубчатой заготовки давлением импульсного магнитного поля // Вестник машиностроения. 2017. № 9. С. 41-43.
6. Грязев М.В., Ларин С.Н., Пасынков А.А. Оценка влияния технологических факторов на предельный коэффициент раздачи трубных заготовок коническим пуансоном // Заготовительные производства в машиностроении. 2017. Т. 15. № 6. С. 255-260.

7. Грязев М.В., Трегубов В.И., Пасынков А.А. Влияние механических свойств на предельные возможности раздачи трубных заготовок коническим пуансоном // Кузнечно-штамповочное производство. Обработка материалов давлением. 2017. № 10. С. 3-7.
8. Подход к оценке напряженно-деформированного состояния трубной заготовки при раздаче коническим пуансоном / Грязев М.В.[ и др.] // Известия Тульского государственного университета. Технические науки. 2017. № 1. 
С. 171-177.
9. Кухарь В.Д., Маленичев Е.С. Разработка математических моделей для анализа процессов магнитно-импульсной штамповки продольных выступов на трубчатых деталях // Известия Тульского государственного университета. Технические науки. 2017. № 1. С. 218-223.
10. Грязев М.В., Ларин С.Н., Пасынков А.А. Подход к оценке напряженно-деформированного состояния трубной заготовки при раздаче коническим пуансоном // Известия Тульского государственного университета. Технические науки. 2017. № 3. С. 3-9.
11. Грязев М.В., Ларин С.Н., Пасынков А.А. Оценка влияния механических свойств трубной заготовки на ее напряженное состояние и силу при раздаче // Известия Тульского государственного университета. Технические науки. 2017. № 6. С. 3-10.
12. Пасынков А.А., Аккуратнова А.С. Оценка напряженно-деформированного состояния и возможностей формоизменения тонкостенной трубной заготовки из сплава ВТ14 при ее раздаче в изотермических условиях // Известия Тульского государственного университета. Технические науки. 2017. № 9-1. С. 286-291.
13. Грязев М.В., Ларин С.Н. Подход к разработке математической модели процесса раздачи трубы коническим пуансоном // Фундаментальные и прикладные проблемы техники и технологии. 2017. № 4-2 (324). С. 12-17.
14. Яковлев С.С., Трегубов В.И., Осипова Е.В. Предельные деформации при ротационной вытяжке с утонением стенки трубных заготовок из анизотропных материалов // Вестник машиностроения. 2016. № 3. С. 81-84.
15. Грязев М.В., Ларин С.Н., Пасынков А.А. Исследование напряженного и деформированного состояния и накопленных микроповреждений при обжиме трубной заготовки в конической матрице // Заготовительные производства в машиностроении. 2016. № 7. С. 13-17.

\clearpage
