%%% Основные сведения %%%
\newcommand{\thesisAuthorLastName}{\fixme{Ступников}}               % Фамилия автора
\newcommand{\thesisAuthorOtherNames}{\fixme{Вадим Владимирович}}    % Полностью Имя и Отчество автора
\newcommand{\thesisAuthorInitials}{\fixme{В.\,В.}}                  % Инициалы автора
\newcommand{\thesisAuthor}                                          % Диссертация, ФИО автора
{%
    \texorpdfstring{                                                % \texorpdfstring takes two arguments and uses the first for (La)TeX and the second for pdf
        \thesisAuthorLastName~\thesisAuthorOtherNames               % так будет отображаться на титульном листе или в тексте, где будет использоваться переменная
    }{%
        \thesisAuthorLastName, \thesisAuthorOtherNames              % эта запись для свойств pdf-файла. В таком виде, если pdf будет обработан программами для сбора библиографических сведений, будет правильно представлена фамилия.
    }
}
\newcommand{\thesisAuthorShort}                                     % Диссертация, ФИО автора инициалами
{\thesisAuthorInitials~\thesisAuthorLastName}
\newcommand{\thesisUdk}                                            % Диссертация, УДК
{\fixme{621.73.043}}
\newcommand{\thesisTitle}                                           %Диссертация, название
{\fixme{Интенсификация технологи получения баллонов высокого давления}}
\newcommand{\thesisSpecialtyNumber}                                 % Диссертация, специальность, номер
{\fixme{05.02.09}}
\newcommand{\thesisSpecialtyTitle}                                  % Диссертация, специальность, название (название взято с сайта ВАК для примера)
{\fixme{Технологии обработки металлов давлением}}
%% \newcommand{\thesisSpecialtyTwoNumber}                           % Диссертация, вторая специальность, номер
%% {\fixme{XX.XX.XX}}
%% \newcommand{\thesisSpecialtyTwoTitle}                            % Диссертация, вторая специальность, название
%% {\fixme{Аддитивные технологии}}
\newcommand{\thesisDegree}                                          % Диссертация, ученая степень
{\fixme{кандидата технических наук}}
\newcommand{\thesisDegreeShort}                                     % Диссертация, ученая степень, краткая запись Для сокращений есть ГОСТы, например: ГОСТ Р 7.0.12-2011 + http://base.garant.ru/179724/#block_30000
{\fixme{к.~т.~н.}}
\newcommand{\thesisCity}                                            % Диссертация, город написания диссертации
{\fixme{Москва}}
\newcommand{\thesisYear}                                            % Диссертация, год написания диссертации
{\the\year}
\newcommand{\thesisOrganization}                                    % Диссертация, организация
{\fixme{Федеральное государственное автономное образовательное учреждение высшего
		образования <<Московский государственный технический университет имени~Н.~Э.~Баумана (национальный исследовательский университет) <<МГТУ им.~Н.~Э.~Баумана>>}}
\newcommand{\thesisOrganizationShort}                               % Диссертация, краткое название организации для доклада
{\fixme{МГТУ им.~Н.~Э.~Баумана}}

\newcommand{\thesisInOrganization}                                  % Диссертация, организация в предложном падеже: Работа выполнена в ...
{\fixme{Московском государственном техническом университете им.~Н.~Э.~Баумана}}

%% \newcommand{\supervisorDead}{}                                   % Рисовать рамку вокруг фамилии
\newcommand{\supervisorFio}                                         % Научный руководитель, ФИО
{\fixme{Евсюков Сергей Александрович}}
\newcommand{\supervisorRegalia}                                     % Научный руководитель, регалии
{\fixme{доктор технических наук, профессор}}
\newcommand{\supervisorFioShort}                                    % Научный руководитель, ФИО
{\fixme{С.\,А.~Евсюков}}
\newcommand{\supervisorRegaliaShort}                                % Научный руководитель, регалии Для сокращений есть ГОСТы, например: ГОСТ Р 7.0.12-2011 + http://base.garant.ru/179724/#block_30000
{\fixme{д.~т.~н.,~проф.}}

%% \newcommand{\supervisorTwoDead}{}                                % Рисовать рамку вокруг фамилии
%% \newcommand{\supervisorTwoFio}                                   % Второй научный руководитель, ФИО
%% {\fixme{Фамилия Имя Отчество}}
%% \newcommand{\supervisorTwoRegalia}                               % Второй научный руководитель, регалии
%% {\fixme{уч. степень, уч. звание}}
%% \newcommand{\supervisorTwoFioShort}                              % Второй научный руководитель, ФИО
%% {\fixme{И.\,О.~Фамилия}}
%% \newcommand{\supervisorTwoRegaliaShort}                          % Второй научный руководитель, регалии
%% {\fixme{уч.~ст.,~уч.~зв.}}

\newcommand{\opponentOneFio}                                        % Оппонент 1, ФИО
{\fixme{Сосенушкин Евгений Наколаевич}}
\newcommand{\opponentOneRegalia}                                    % Оппонент 1, регалии
{\fixme{доктор технических наук, профессор}}
\newcommand{\opponentOneJobPlace}                                   % Оппонент 1, место работы
{\fixme{кафедра систем пластического деформирования ФГБОУ ВО <<МГТУ "СТАНКИН">>}}
\newcommand{\opponentOneJobPost}                                    % Оппонент 1, должность
{\fixme{старший научный сотрудник}}

\newcommand{\opponentTwoFio}                                        % Оппонент 2, ФИО
{\fixme{Фамилия Имя Отчество}}
\newcommand{\opponentTwoRegalia}                                    % Оппонент 2, регалии
{\fixme{кандидат физико-математических наук}}
\newcommand{\opponentTwoJobPlace}                                   % Оппонент 2, место работы
{\fixme{Основное место работы c длинным длинным длинным длинным названием}}
\newcommand{\opponentTwoJobPost}                                    % Оппонент 2, должность
{\fixme{старший научный сотрудник}}

%% \newcommand{\opponentThreeFio}                                   % Оппонент 3, ФИО
%% {\fixme{Фамилия Имя Отчество}}
%% \newcommand{\opponentThreeRegalia}                               % Оппонент 3, регалии
%% {\fixme{кандидат физико-математических наук}}
%% \newcommand{\opponentThreeJobPlace}                              % Оппонент 3, место работы
%% {\fixme{Основное место работы c длинным длинным длинным длинным названием}}
%% \newcommand{\opponentThreeJobPost}                               % Оппонент 3, должность
%% {\fixme{старший научный сотрудник}}

\newcommand{\leadingOrganizationTitle}                              % Ведущая организация, дополнительные строки. Удалить, чтобы не отображать в автореферате
{\fixme{АО~<<НПО~Энергомаш>>}}

\newcommand{\defenseDate}                                           % Защита, дата
{\fixme{01 декабря 2024~г.~в~11 часов}}
\newcommand{\defenseCouncilNumber}                                  % Защита, номер диссертационного совета
%{\fixme{Д\,212.141.04}}
{\fixme{24.2.331.01}}
\newcommand{\defenseCouncilTitle}                                   % Защита, учреждение диссертационного совета
{\fixme{МГТУ им.~Н.~Э.~Баумана}}
\newcommand{\defenseCouncilAddress}                                 % Защита, адрес учреждение диссертационного совета
{\fixme{105005,~г. Москва, ул. 2---я Бауманская, д.~5, стр.~1}}
\newcommand{\defenseCouncilPhone}                                   % Телефон для справок
{\fixme{8~(499)~267-09-63}}

\newcommand{\defenseSecretaryFio}                                   % Секретарь диссертационного совета, ФИО
{\fixme{Плохих Андрей Иванович}}
\newcommand{\defenseSecretaryRegalia}                               % Секретарь диссертационного совета, регалии
{\fixme{к.~т.~н, доцент}}                                           % Для сокращений есть ГОСТы, например: ГОСТ Р 7.0.12-2011 + http://base.garant.ru/179724/#block_30000

\newcommand{\synopsisLibrary}                                       % Автореферат, название библиотеки
{\fixme{МГТУ им.~Н.~Э.~Баумана}}
\newcommand{\synopsisDate}                                          % Автореферат, дата рассылки
{\fixme{01 декабря}\the\year~года}


\providecommand{\keywords}                                          % Ключевые слова для метаданных PDF диссертации и автореферата
{вытяжка, обжим, AutoForm, конечно-элементное моделирование,FEM, штамповка, обработка металлов давлением, толщина, утонение, масса}
