%%% Основные сведения %%%
\newcommand{\thesisAuthorLastName}{\fixme{Ступников}}
\newcommand{\thesisAuthorLastNameTo}{\fixme{Ступникову}}
\newcommand{\thesisAuthorFirstName}{\fixme{Вадим}}
\newcommand{\thesisAuthorFirstNameTo}{\fixme{Вадиму}}
\newcommand{\thesisAuthorSurName}{\fixme{Владимирович}}
\newcommand{\thesisAuthorSurNameTo}{\fixme{Владимировичу}}
\newcommand{\thesisAuthorOtherNames}{\fixme{Вадим Владимирович}}
\newcommand{\thesisAuthorInitials}{\fixme{В.\,В.}}
\newcommand{\thesisAuthor}             % Диссертация, ФИО автора
{%
    \texorpdfstring{% \texorpdfstring takes two arguments and uses the first for (La)TeX and the second for pdf
        \thesisAuthorLastName~\thesisAuthorOtherNames% так будет отображаться на титульном листе или в тексте, где будет использоваться переменная
    }{%
        \thesisAuthorLastName, \thesisAuthorOtherNames% эта запись для свойств pdf-файла. В таком виде, если pdf будет обработан программами для сбора библиографических сведений, будет правильно представлена фамилия.
    }
}
\newcommand{\thesisAuthorShort}        % Диссертация, ФИО автора инициалами
{\thesisAuthorInitials~\thesisAuthorLastName}
\newcommand{\thesisUdk}                % Диссертация, УДК
{\fixme{621.73.043}}
\newcommand{\thesisTitle}              % Диссертация, название
{\fixme{Совершенствование технологического процесса обжима в сферическую матрицу}}
\newcommand{\thesisSpecialtyNumber}    % Диссертация, специальность, номер
{\fixme{05.02.09}}
\newcommand{\thesisSpecialtyTitle}     % Диссертация, специальность, название (название взято с сайта ВАК для примера)
{\fixme{Технологии и машины обработки давлением}}
%% \newcommand{\thesisSpecialtyTwoNumber} % Диссертация, вторая специальность, номер
%% {\fixme{XX.XX.XX}}
%% \newcommand{\thesisSpecialtyTwoTitle}  % Диссертация, вторая специальность, название
%% {\fixme{Теория и~методика физического воспитания, спортивной тренировки, оздоровительной и~адаптивной физической культуры}}
\newcommand{\thesisDegree}             % Диссертация, ученая степень
{\fixme{кандидата технических наук}}
\newcommand{\thesisDegreeShort}        % Диссертация, ученая степень, краткая запись
{\fixme{к.~т.~н.}}
\newcommand{\thesisCity}               % Диссертация, город написания диссертации
{\fixme{Москва}}
\newcommand{\thesisYear}               % Диссертация, год написания диссертации
{\the\year}
\newcommand{\thesisOrganization}       % Диссертация, организация
{\fixme{Федеральное государственное бюджетное образовательное учреждение высшего образования <<Московский государственный технический университет имени Н.~Э.~Баумана (национальный исследовательский университет)>> (МГТУ им.~Н.~Э.~Баумана)}}
\newcommand{\thesisOrganizationShort}  % Диссертация, краткое название организации для доклада
{\fixme{\mbox{МГТУ им.~Н.~Э.~Баумана}}}

\newcommand{\thesisInOrganization}     % Диссертация, организация в предложном падеже: Работа выполнена в ...
{\fixme{Московском государственном техническом университете им.~Н.~Э.~Баумана}}
\newcommand{\thesisWhereOrganization}     % Диссертация, организация в предложном падеже: Работа выполнена в ...
{\fixme{Московского государственного технического университета им.~Н.~Э.~Баумана}}
\newcommand{\thesisInOrganizationFull}     % Диссертация, организация в предложном падеже: Работа выполнена в ...
{\fixme{федеральном государственном бюджетном образовательном учреждении высшего образования  <<Московский государственный технический университет имени Н.~Э.~Баумана (национальный исследовательский университет)>> (МГТУ им.~Н.~Э.~Баумана)}}
\newcommand{\thesisFromOrganizationFull}     % Диссертация, организация в предложном падеже: Работа выполнена в ...
{\fixme{федерального государственного бюджетного образовательного учреждения высшего образования  <<Московский государственный технический университет имени Н.~Э.~Баумана (национальный исследовательский университет)>> (МГТУ им.~Н.~Э.~Баумана)}}

%% \newcommand{\supervisorDead}{}           % Рисовать рамку вокруг фамилии
\newcommand{\supervisorFio}              % Научный руководитель, ФИО
{\fixme{Евсюков Сергей Александрович}}
\newcommand{\supervisorRegalia}          % Научный руководитель, регалии
{\fixme{доктор технических наук, профессор}}
\newcommand{\supervisorFioShort}         % Научный руководитель, ФИО
{\fixme{С.\,А.~Евсюков}}
\newcommand{\supervisorRegaliaShort}     % Научный руководитель, регалии
{\fixme{д.~т.~н.,~проф.}}

%% \newcommand{\supervisorTwoDead}{}        % Рисовать рамку вокруг фамилии
%% \newcommand{\supervisorTwoFio}           % Второй научный руководитель, ФИО
%% {\fixme{Фамилия Имя Отчество}}
%% \newcommand{\supervisorTwoRegalia}       % Второй научный руководитель, регалии
%% {\fixme{уч. степень, уч. звание}}
%% \newcommand{\supervisorTwoFioShort}      % Второй научный руководитель, ФИО
%% {\fixme{И.\,О.~Фамилия}}
%% \newcommand{\supervisorTwoRegaliaShort}  % Второй научный руководитель, регалии
%% {\fixme{уч.~ст.,~уч.~зв.}}

\newcommand{\opponentOneFio}           % Оппонент 1, ФИО
{\fixme{~~~~~~~~~~~~~~~~~~~~~~~}}
%\newcommand{\opponentOneFio}           % Оппонент 1, ФИО
%{\fixme{Вдовин Сергей Иванович}}
\newcommand{\opponentOneRegalia}       % Оппонент 1, регалии
{\fixme{доктор технических наук, профессор}}
\newcommand{\opponentOneJobPlace}      % Оппонент 1, место работы
{\fixme{кафедра систем пластического деформирования ФГБОУ ВО <<МГТУ "СТАНКИН">>}}
\newcommand{\opponentOneJobPlaceOld}      % Оппонент 1, место работы бывшее (больше не работает)
{\fixme{<<Машиностроения>> ФГБОУ ВО <<ОГУ имени И.С. Тургенева>>}}

\newcommand{\opponentOneJobPost}       % Оппонент 1, должность
{\fixme{старший научный сотрудник}}

\newcommand{\opponentTwoFio}           % Оппонент 2, ФИО
{\fixme{~~~~~~~~~~~~~~~~~~~~~~~}}
%\newcommand{\opponentTwoFio}           % Оппонент 2, ФИО
%{\fixme{Шпунькин Николай Фомич}}
\newcommand{\opponentTwoRegalia}       % Оппонент 2, регалии
{\fixme{кандидат технических наук, доцент}}
\newcommand{\opponentTwoJobPlace}      % Оппонент 2, место работы
{\fixme{<<Обработка материалов давлением и аддитивные технологии>> ФГБОУ ВО <<Московский политехнический университет>>}}
\newcommand{\opponentTwoJobPost}       % Оппонент 2, должность
{\fixme{профессор}}

%% \newcommand{\opponentThreeFio}         % Оппонент 3, ФИО
%% {\fixme{Фамилия Имя Отчество}}
%% \newcommand{\opponentThreeRegalia}     % Оппонент 3, регалии
%% {\fixme{кандидат физико-математических наук}}
%% \newcommand{\opponentThreeJobPlace}    % Оппонент 3, место работы
%% {\fixme{Основное место работы c длинным длинным длинным длинным названием}}
%% \newcommand{\opponentThreeJobPost}     % Оппонент 3, должность
%% {\fixme{старший научный сотрудник}}

\newcommand{\leadingOrganizationTitle} % Ведущая организация, дополнительные строки. Удалить, чтобы не отображать в автореферате
{\fixme{ФГБОУ ВО <<Тульский государственный университет>>}}
%{\fixme{АО~<<НПО~Энергомаш>>}} % Старое название, организация где было внедрение
\newcommand{\implementationOrganizationTitle} % Организация где происходит внедрение
{\fixme{АО~<<НПО~Энергомаш>>}}

\newcommand{\defenseDate}              % Защита, дата
{\fixme{<<~~~>>~~~~~~~}\the\year~г.~в~~~~~ч.~~~~~мин}
\newcommand{\defenseCouncilNumber}     % Защита, номер диссертационного совета
%{\fixme{Д\,212.}}
{\fixme{24.2.331.01}}
\newcommand{\defenseCouncilTitle}      % Защита, учреждение диссертационного совета
{\fixme{МГТУ~им.~Н.~Э.~Баумана}}
\newcommand{\defenseCouncilAddress}    % Защита, адрес учреждение диссертационного совета
{\fixme{105005,~г. Москва, ул. 2---я Бауманская, д.~5, стр.~1}}
\newcommand{\defenseCouncilPhone}      % Телефон для справок
{\fixme{(499)~267-09-63}}

\newcommand{\defenseSecretaryFioShort}      % Секретарь диссертационного совета, ФИО
{\fixme{Плохих А.И.}}
\newcommand{\defenseSecretaryFio}      % Секретарь диссертационного совета, ФИО
{\fixme{Плохих Андрей Иванович}}
\newcommand{\defenseSecretaryRegalia}  % Секретарь диссертационного совета, регалии
{\fixme{к.~т.~н, доцент}}            % Для сокращений есть ГОСТы, например: ГОСТ Р 7.0.12-2011 + http://base.garant.ru/179724/#block_30000
\newcommand{\defenseSecretaryRegaliaFull}  % Секретарь диссертационного совета, регалии
{\fixme{кандидат технических наук, доцент}} 

\newcommand{\synopsisLibrary}          % Автореферат, название библиотеки
{\fixme{МГТУ им.~Н.~Э.~Баумана}}
\newcommand{\synopsisDate}             % Автореферат, дата рассылки
{\fixme{<<~~~>>~~~~~~~}\the\year~г.}


% Счетчики Публикаций
\newcommand{\publicationsAllDisser}             % Диссертация состоит из .... и списка литературы из ??? наименований
{\fixme{160}}
\newcommand{\publicationsAll}             % Основное содержание диссертационной работы изложено в научных работах
{\fixme{9}}
\newcommand{\publicationsAllAll}             % Всего работ
{\fixme{21}}
% Публикации
\newcommand{\publicationsAllVak}             % в том числе 3 изданиях, рекомендованных ВАК РФ
{\fixme{5}}
\newcommand{\publicationsAllVakV}            % в том числе 3 изданиях, рекомендованных ВАК РФ, общим объемом 5,86 печ. л.
{\fixme{9,86 печ. л.}}
\newcommand{\patentAll}            % Зарегистрировано патента
{\fixme{3}}

% Счетчики Диссертации. Текст диссертации содержит 148 машинописных страниц, включая 32 таблицы и 79 рисунков.
\newcommand{\dissertationCountPages}             % Текст диссертации содержит 148 машинописных страниц
{\fixme{148}}
\newcommand{\dissertationCountTables}             % Текст диссертации содержит 32 таблицы 
{\fixme{32}}
\newcommand{\dissertationCountPics}             % Текст диссертации содержит 79 рисунков.
{\fixme{79}}


% To avoid conflict with beamer class use \providecommand
\providecommand{\keywords}%            % Ключевые слова для метаданных PDF диссертации и автореферата
{вытяжка, обжим, AutoForm, конечно-элементное моделирование, FEM, штамповка, обработка металлов давлением, толщина, утонение, масса}
